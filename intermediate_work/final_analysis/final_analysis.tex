% Options for packages loaded elsewhere
\PassOptionsToPackage{unicode}{hyperref}
\PassOptionsToPackage{hyphens}{url}
%
\documentclass[
]{article}
\usepackage{amsmath,amssymb}
\usepackage{iftex}
\ifPDFTeX
  \usepackage[T1]{fontenc}
  \usepackage[utf8]{inputenc}
  \usepackage{textcomp} % provide euro and other symbols
\else % if luatex or xetex
  \usepackage{unicode-math} % this also loads fontspec
  \defaultfontfeatures{Scale=MatchLowercase}
  \defaultfontfeatures[\rmfamily]{Ligatures=TeX,Scale=1}
\fi
\usepackage{lmodern}
\ifPDFTeX\else
  % xetex/luatex font selection
\fi
% Use upquote if available, for straight quotes in verbatim environments
\IfFileExists{upquote.sty}{\usepackage{upquote}}{}
\IfFileExists{microtype.sty}{% use microtype if available
  \usepackage[]{microtype}
  \UseMicrotypeSet[protrusion]{basicmath} % disable protrusion for tt fonts
}{}
\makeatletter
\@ifundefined{KOMAClassName}{% if non-KOMA class
  \IfFileExists{parskip.sty}{%
    \usepackage{parskip}
  }{% else
    \setlength{\parindent}{0pt}
    \setlength{\parskip}{6pt plus 2pt minus 1pt}}
}{% if KOMA class
  \KOMAoptions{parskip=half}}
\makeatother
\usepackage{xcolor}
\usepackage[margin=1in]{geometry}
\usepackage{color}
\usepackage{fancyvrb}
\newcommand{\VerbBar}{|}
\newcommand{\VERB}{\Verb[commandchars=\\\{\}]}
\DefineVerbatimEnvironment{Highlighting}{Verbatim}{commandchars=\\\{\}}
% Add ',fontsize=\small' for more characters per line
\usepackage{framed}
\definecolor{shadecolor}{RGB}{248,248,248}
\newenvironment{Shaded}{\begin{snugshade}}{\end{snugshade}}
\newcommand{\AlertTok}[1]{\textcolor[rgb]{0.94,0.16,0.16}{#1}}
\newcommand{\AnnotationTok}[1]{\textcolor[rgb]{0.56,0.35,0.01}{\textbf{\textit{#1}}}}
\newcommand{\AttributeTok}[1]{\textcolor[rgb]{0.13,0.29,0.53}{#1}}
\newcommand{\BaseNTok}[1]{\textcolor[rgb]{0.00,0.00,0.81}{#1}}
\newcommand{\BuiltInTok}[1]{#1}
\newcommand{\CharTok}[1]{\textcolor[rgb]{0.31,0.60,0.02}{#1}}
\newcommand{\CommentTok}[1]{\textcolor[rgb]{0.56,0.35,0.01}{\textit{#1}}}
\newcommand{\CommentVarTok}[1]{\textcolor[rgb]{0.56,0.35,0.01}{\textbf{\textit{#1}}}}
\newcommand{\ConstantTok}[1]{\textcolor[rgb]{0.56,0.35,0.01}{#1}}
\newcommand{\ControlFlowTok}[1]{\textcolor[rgb]{0.13,0.29,0.53}{\textbf{#1}}}
\newcommand{\DataTypeTok}[1]{\textcolor[rgb]{0.13,0.29,0.53}{#1}}
\newcommand{\DecValTok}[1]{\textcolor[rgb]{0.00,0.00,0.81}{#1}}
\newcommand{\DocumentationTok}[1]{\textcolor[rgb]{0.56,0.35,0.01}{\textbf{\textit{#1}}}}
\newcommand{\ErrorTok}[1]{\textcolor[rgb]{0.64,0.00,0.00}{\textbf{#1}}}
\newcommand{\ExtensionTok}[1]{#1}
\newcommand{\FloatTok}[1]{\textcolor[rgb]{0.00,0.00,0.81}{#1}}
\newcommand{\FunctionTok}[1]{\textcolor[rgb]{0.13,0.29,0.53}{\textbf{#1}}}
\newcommand{\ImportTok}[1]{#1}
\newcommand{\InformationTok}[1]{\textcolor[rgb]{0.56,0.35,0.01}{\textbf{\textit{#1}}}}
\newcommand{\KeywordTok}[1]{\textcolor[rgb]{0.13,0.29,0.53}{\textbf{#1}}}
\newcommand{\NormalTok}[1]{#1}
\newcommand{\OperatorTok}[1]{\textcolor[rgb]{0.81,0.36,0.00}{\textbf{#1}}}
\newcommand{\OtherTok}[1]{\textcolor[rgb]{0.56,0.35,0.01}{#1}}
\newcommand{\PreprocessorTok}[1]{\textcolor[rgb]{0.56,0.35,0.01}{\textit{#1}}}
\newcommand{\RegionMarkerTok}[1]{#1}
\newcommand{\SpecialCharTok}[1]{\textcolor[rgb]{0.81,0.36,0.00}{\textbf{#1}}}
\newcommand{\SpecialStringTok}[1]{\textcolor[rgb]{0.31,0.60,0.02}{#1}}
\newcommand{\StringTok}[1]{\textcolor[rgb]{0.31,0.60,0.02}{#1}}
\newcommand{\VariableTok}[1]{\textcolor[rgb]{0.00,0.00,0.00}{#1}}
\newcommand{\VerbatimStringTok}[1]{\textcolor[rgb]{0.31,0.60,0.02}{#1}}
\newcommand{\WarningTok}[1]{\textcolor[rgb]{0.56,0.35,0.01}{\textbf{\textit{#1}}}}
\usepackage{longtable,booktabs,array}
\usepackage{calc} % for calculating minipage widths
% Correct order of tables after \paragraph or \subparagraph
\usepackage{etoolbox}
\makeatletter
\patchcmd\longtable{\par}{\if@noskipsec\mbox{}\fi\par}{}{}
\makeatother
% Allow footnotes in longtable head/foot
\IfFileExists{footnotehyper.sty}{\usepackage{footnotehyper}}{\usepackage{footnote}}
\makesavenoteenv{longtable}
\usepackage{graphicx}
\makeatletter
\def\maxwidth{\ifdim\Gin@nat@width>\linewidth\linewidth\else\Gin@nat@width\fi}
\def\maxheight{\ifdim\Gin@nat@height>\textheight\textheight\else\Gin@nat@height\fi}
\makeatother
% Scale images if necessary, so that they will not overflow the page
% margins by default, and it is still possible to overwrite the defaults
% using explicit options in \includegraphics[width, height, ...]{}
\setkeys{Gin}{width=\maxwidth,height=\maxheight,keepaspectratio}
% Set default figure placement to htbp
\makeatletter
\def\fps@figure{htbp}
\makeatother
\setlength{\emergencystretch}{3em} % prevent overfull lines
\providecommand{\tightlist}{%
  \setlength{\itemsep}{0pt}\setlength{\parskip}{0pt}}
\setcounter{secnumdepth}{-\maxdimen} % remove section numbering
\usepackage{adjustbox}
\ifLuaTeX
  \usepackage{selnolig}  % disable illegal ligatures
\fi
\usepackage{bookmark}
\IfFileExists{xurl.sty}{\usepackage{xurl}}{} % add URL line breaks if available
\urlstyle{same}
\hypersetup{
  hidelinks,
  pdfcreator={LaTeX via pandoc}}

\author{}
\date{\vspace{-2.5em}}

\begin{document}

\begin{titlepage}
    \centering
    {\Huge \textbf{Stress, Sleep, Emotion Regulation, and Support-Seeking: Insights from a Large University Mental-Health Survey}\par}
    \vspace{2cm}

    {\Large\textbf{Group Members:}}\par
    {\large Andy Chen \\ Krzysztof Oleksiak \\ Rafael Serrano Aparicio \\ Jeffrey Yan Man Fong \\ Danny Han \par}

    \vfill
    {\large \today\par}
\end{titlepage}

\newpage

\section{Abstract}\label{abstract}

This project analyzes data from the University Student Mental Health
survey, a large-scale study of Canadian undergraduate students collected
in Fall 2020 to measure psychological wellbeing, stress, emotional
regulation, sleep behaviours, and social support. Using validated
scales---including the DERS-16 for emotional regulation, the Seeking
Social Support scale, the DASS-21, and the Perceived Stress Scale---we
examined four key relationships: differences in emotional-regulation
difficulties across academic programs, differences in social-support
seeking across programs, the association between restfulness and
mental-health symptoms, and changes in perceived stress before versus
after the COVID-19 outbreak. Our analyses revealed minimal or
non-significant differences across programs for both
emotional-regulation difficulties and social-support seeking, a
meaningful link between feeling well-rested and lower DASS symptom
scores, and a significant increase in perceived stress following the
onset of COVID-19. These findings suggest that individual wellbeing
factors and external stressors may exert stronger influences on student
mental health than academic program alone.

\section{Introduction}\label{introduction}

The \emph{University Student Mental Health} dataset contains survey
responses from 1,192 undergraduate students across Canada, collected
between September 22 and October 30, 2020. Recruitment occurred online
through Facebook university groups, Reddit, Instagram, Twitter, and
research-participant portals such as REACH BC and the CPA R2P2 portal.
Participants completed a Qualtrics survey designed to assess multiple
aspects of student wellbeing during the COVID-19 pandemic, including
emotional regulation, perceived stress, depression--anxiety--stress
symptoms, sleep behaviour, social-support seeking, physical activity,
and demographic context. The dataset includes 147 variables spanning
validated psychological scales, behavioural measures, and demographic
information, offering a rich snapshot of undergraduate wellbeing during
a period of significant external stress.

The purpose of this project is to investigate four key research
questions regarding student mental health:

\begin{enumerate}
\def\labelenumi{\arabic{enumi}.}
\tightlist
\item
  \textbf{Is there a correlation between the Program of Study and the
  responses to the Difficulties in Emotional Regulation Scale (DERS)
  Questionnaire?}
\item
  \textbf{Do students in different academic programs differ in how much
  social support they seek?}
\item
  \textbf{Is feeling well-rested associated with differences in
  mental-health symptoms measured by the DASS-21?}
\item
  \textbf{Did students' perceived stress levels change significantly
  from before to after the COVID-19 outbreak, based on pre/post PSS
  scores?}
\item
  \textbf{Is there a relationship between students' emotional-regulation
  difficulties (DERS) and their hobby patterns, including both the
  importance they place on different hobbies and the amount of time they
  spend on them?}
\end{enumerate}

A focused subset of variables was selected from the full dataset, and
each variable was cleaned, recoded, and scored according to the
documentation provided by the dataset creators. Program of Study was
treated as a categorical factor, originally coded 1--8 and recoded into
descriptive program labels. The \textbf{Rested} variable was treated as
an ordered factor with levels ``Yes,'' ``Somewhat,'' and ``No.''
Psychological scales were scored using their validated scoring
procedures: DERS-16 items were summed, SSS items were averaged, DASS-21
items were summed and doubled for subscale totals, and PSS items were
summed with designated items reverse-scored. Missing data were handled
using listwise deletion for the analyses. All scoring procedures
followed the official scale documentation.

\newpage

\subsubsection{Mini Codebook}\label{mini-codebook}

\begin{table}[h!]
\centering
\small
\begin{tabular}{p{2.6cm} p{3.8cm} p{3.0cm} p{4.0cm} p{3.3cm}}
\toprule
\textbf{Variable} & \textbf{Description} & \textbf{Coding / Scale} & \textbf{Recoding / Scoring} & \textbf{Missing Treatment} \\
\midrule
Program & Student’s academic program of study & 1–8 coded as program categories & Recoded into descriptive labels (e.g., Business, Education, Engineering, etc.) & Listwise deletion \\
DERS\_1--DERS\_16 & Difficulties in Emotional Regulation Scale items & 5-point Likert (1–5) & Averaged into DERS\_mean score; higher = more dysregulation & Listwise deletion \\
SSS\_1--SSS\_12 & Seeking Social Support scale & 5-point Likert (1–5) & Averaged into total SSS score & Listwise deletion \\
Rested & Self-reported restfulness upon waking & 1 = Yes, 2 = Somewhat, 3 = No & Treated as ordered factor & Listwise deletion \\
DASS\_1--DASS\_21 & Depression Anxiety Stress Scale & 4-point Likert (0–3) & Summed and doubled for subscales; overall DASS average computed & Listwise deletion \\
Pre\_PSS\_1--Pre\_PSS\_10 & Perceived Stress Scale (before COVID-19) & 0–4 Likert & Items 4, 5, 7, 8 reverse-scored; summed into total & Listwise deletion \\
Post\_PSS\_1--Post\_PSS\_10 & Perceived Stress Scale (after COVID-19) & 0–4 Likert & Items 4, 5, 7, 8 reverse-scored; summed into total & Listwise deletion \\
Hobbies\_Imp\_1--Hobbies\_Imp\_8 & Importance of various hobbies & 5-point Likert (1–5) & Averaged into Imp\_overall; also used individually & Listwise deletion \\
Hobbies\_Time\_1--Hobbies\_Time\_8 & Weekly time spent on hobbies & 7-point Likert (1–7) & Averaged into Time\_overall; also used individually & Listwise deletion \\
\bottomrule
\end{tabular}
\end{table}

\newpage

\section{Research Question 1: Program of Study and Emotional Regulation
(DERS)}\label{research-question-1-program-of-study-and-emotional-regulation-ders}

This section examines whether emotional-regulation difficulties differ
across academic programs. The DERS-16 scale measures how frequently
students experience problems with emotional regulation, with higher
scores reflecting more frequent difficulties. Each participant's average
DERS score was computed by taking the mean across the 16 DERS items.

\subsubsection{Distribution of DERS Scores Across
Programs}\label{distribution-of-ders-scores-across-programs}

\includegraphics{final_analysis_files/figure-latex/unnamed-chunk-2-1.pdf}
The distribution of DERS scores appears broadly similar across programs,
though Education shows a slightly higher median compared to others. The
overlap suggests differences may be modest, requiring statistical
testing.

\subsubsection{ANOVA Test}\label{anova-test}

\begin{verbatim}
##              Df Sum Sq Mean Sq F value Pr(>F)   
## Program       7   17.6  2.5153   3.039 0.0037 **
## Residuals   740  612.4  0.8276                  
## ---
## Signif. codes:  0 '***' 0.001 '**' 0.01 '*' 0.05 '.' 0.1 ' ' 1
\end{verbatim}

The one-way ANOVA indicates statistically significant differences among
at least some program means (α = 0.05). This suggests that average
emotional-regulation difficulty varies slightly across academic fields,
though the effect appears small.

\subsubsection{Confidence Intervals for Mean DERS
Scores}\label{confidence-intervals-for-mean-ders-scores}

\includegraphics{final_analysis_files/figure-latex/unnamed-chunk-4-1.pdf}
Programs show substantial overlap in confidence intervals, reinforcing
that while differences exist, they are relatively small in magnitude.

\subsubsection{Summary Statistics}\label{summary-statistics}

\begin{verbatim}
##    Min. 1st Qu.  Median    Mean 3rd Qu.    Max. 
##   1.000   2.188   2.875   2.894   3.578   5.000
\end{verbatim}

\begin{verbatim}
##    Min. 1st Qu.  Median    Mean 3rd Qu.    Max. 
##   1.000   2.000   3.000   2.901   4.000   5.000
\end{verbatim}

\subsubsection{Interpretation}\label{interpretation}

The mean DERS score across students lies between 2 and 3, meaning the
average respondent experiences emotional-regulation difficulties
somewhere between ``sometimes'' and ``about half the time.'' The median
DERS response (3) indicates that half of all students report moderate
emotional-regulation difficulty (36\%--65\% of the time). Overall, the
data suggest that emotional-regulation challenges are fairly common
across programs.

\newpage

\section{Research Question 2: Program of Study and Seeking Social
Support
(SSS)}\label{research-question-2-program-of-study-and-seeking-social-support-sss}

This analysis investigates whether the amount of social support students
seek differs across academic programs. The Seeking Social Support (SSS)
scale is a 12-item instrument scored on a 1--5 Likert scale, where
higher values indicate greater levels of support-seeking behavior. For
each participant, we calculated the mean SSS score across all items.

\subsubsection{Distribution of DERS Scores Across
Programs}\label{distribution-of-ders-scores-across-programs-1}

\includegraphics{final_analysis_files/figure-latex/unnamed-chunk-7-1.pdf}
The distribution of DERS scores appears broadly similar across programs,
though Education shows a slightly higher median compared to others. The
overlap suggests differences may be modest, requiring statistical
testing.

\subsubsection{ANOVA Test}\label{anova-test-1}

\begin{verbatim}
## 
##  One-way analysis of means (not assuming equal variances)
## 
## data:  SSS and Program
## F = 1.5207, num df = 7.00, denom df = 102.57, p-value = 0.1685
\end{verbatim}

The p-value is large, indicating no significant differences in SSS
between academic programs.

\subsubsection{SSS Differences by Sex}\label{sss-differences-by-sex}

\includegraphics{final_analysis_files/figure-latex/unnamed-chunk-9-1.pdf}
Female and male students show slightly different distributions, but the
overlap is substantial.

\subsubsection{t-Test: Female vs Male}\label{t-test-female-vs-male}

\begin{verbatim}
## 
##  Welch Two Sample t-test
## 
## data:  survey.data$SSS[survey.data$Sex == "Female"] and survey.data$SSS[survey.data$Sex == "Male"]
## t = 1.968, df = 116.74, p-value = 0.05144
## alternative hypothesis: true difference in means is not equal to 0
## 95 percent confidence interval:
##  -0.001005742  0.317077317
## sample estimates:
## mean of x mean of y 
##  3.300509  3.142473
\end{verbatim}

The p-value is just above 0.05. This suggests no statistically reliable
difference in SSS between men and women.

\subsubsection{SSS Differences by International
Status}\label{sss-differences-by-international-status}

\includegraphics{final_analysis_files/figure-latex/unnamed-chunk-11-1.pdf}
International and domestic students appear very similar in how much
social support they report seeking.

\subsubsection{Relationship Between SSS and Emotional-Regulation
Difficulty
(DERS)}\label{relationship-between-sss-and-emotional-regulation-difficulty-ders}

\includegraphics{final_analysis_files/figure-latex/unnamed-chunk-12-1.pdf}
There is a slight downward pattern: higher DERS (more
emotional-regulation difficulty) is associated with lower SSS.

\subsubsection{High vs Low DERS Groups}\label{high-vs-low-ders-groups}

We divide students into two groups based on whether their DERS score
exceeds 64.

\includegraphics{final_analysis_files/figure-latex/unnamed-chunk-13-1.pdf}

Students with very high DERS scores clearly show lower SSS values.

\subsubsection{t-Test: Low DERS vs High
DERS}\label{t-test-low-ders-vs-high-ders}

\begin{verbatim}
## 
##  Welch Two Sample t-test
## 
## data:  survey.data$SSS[survey.data$DERS <= 64] and survey.data$SSS[survey.data$DERS > 64]
## t = 5.0255, df = 117.47, p-value = 1.816e-06
## alternative hypothesis: true difference in means is not equal to 0
## 95 percent confidence interval:
##  0.2425650 0.5580672
## sample estimates:
## mean of x mean of y 
##  3.331167  2.930851
\end{verbatim}

The p-value is extremely small (1.8e-06). This indicates a highly
significant difference: Students who struggle more with emotional
regulation tend to seek less social support.

\subsubsection{Interpretation}\label{interpretation-1}

Across academic programs, sex, and international status, students report
similar levels of social-support seeking. The only meaningful predictor
in this analysis is emotional-regulation difficulty:

Students with high DERS scores consistently report lower SSS.

This suggests that the tendency to seek social support is influenced
more by psychological factors than by demographic or academic ones.

\newpage

\section{Research Question 3: Restfulness and Mental-Health Symptoms
(DASS-21)}\label{research-question-3-restfulness-and-mental-health-symptoms-dass-21}

This analysis examines whether students who report feeling well-rested
differ in their mental-health symptoms, as measured by the DASS-21
(Depression, Anxiety, and Stress Scale). Each student answered 21 items
scored from 0--3, and higher average scores indicate greater
psychological distress. The Rested variable contains three groups:
\textbf{Yes (1)}, \textbf{Somewhat (2)}, and \textbf{No (3)}.

\begin{longtable}[]{@{}lrrr@{}}
\caption{Average DASS Scores by Rested Group}\tabularnewline
\toprule\noalign{}
DASS\_Variable & 1 & 2 & 3 \\
\midrule\noalign{}
\endfirsthead
\toprule\noalign{}
DASS\_Variable & 1 & 2 & 3 \\
\midrule\noalign{}
\endhead
\bottomrule\noalign{}
\endlastfoot
n & 155.000000 & 387.000000 & 206.000000 \\
DASS\_1 & 1.974193 & 2.333333 & 2.757282 \\
DASS\_2 & 1.793548 & 1.956072 & 2.150485 \\
DASS\_3 & 1.625806 & 1.878553 & 2.378641 \\
DASS\_4 & 1.470968 & 1.728682 & 1.975728 \\
DASS\_5 & 2.380645 & 2.764858 & 3.310680 \\
DASS\_6 & 1.993548 & 2.121447 & 2.461165 \\
DASS\_7 & 1.574194 & 1.829457 & 2.082524 \\
DASS\_8 & 2.161290 & 2.307494 & 2.762136 \\
DASS\_9 & 1.896774 & 2.180879 & 2.587379 \\
DASS\_10 & 1.812903 & 2.147287 & 2.718447 \\
DASS\_11 & 2.077419 & 2.320413 & 2.742718 \\
DASS\_12 & 2.103226 & 2.529716 & 3.077670 \\
DASS\_13 & 2.070968 & 2.372093 & 2.936893 \\
DASS\_14 & 1.864516 & 1.989664 & 2.296117 \\
DASS\_15 & 1.741936 & 1.956072 & 2.514563 \\
DASS\_16 & 1.774193 & 2.033592 & 2.621359 \\
DASS\_17 & 1.787097 & 1.935401 & 2.480583 \\
DASS\_18 & 1.767742 & 1.852713 & 2.208738 \\
DASS\_19 & 1.580645 & 1.863049 & 2.189320 \\
DASS\_20 & 1.741936 & 1.860465 & 2.223301 \\
DASS\_21 & 1.632258 & 1.780362 & 2.330097 \\
Overall\_DASS\_Average & 1.848848 & 2.082933 & 2.514563 \\
\end{longtable}

Students who report being well-rested exhibit clearly lower average DASS
scores, indicating fewer symptoms of depression, anxiety, and stress.
Students who report not feeling rested show the highest levels of
psychological distress, with the ``Somewhat'' group in the middle.

\subsubsection{Distribution of DASS Scores by Rested
Group}\label{distribution-of-dass-scores-by-rested-group}

\includegraphics{final_analysis_files/figure-latex/unnamed-chunk-16-1.pdf}
The boxplot shows a clear trend:

\begin{itemize}
\item
  Well-rested students: lowest median DASS, smallest variability
\item
  Somewhat rested students: moderate DASS
\item
  Not rested students: highest median and widest spread
\end{itemize}

The upward shift of medians and increasing variability from left to
right strongly suggests a relationship between restfulness and
mental-health symptoms.

\subsubsection{ANOVA Test}\label{anova-test-2}

\begin{verbatim}
##             Df Sum Sq Mean Sq F value   Pr(>F)    
## Rested       2  4.790  2.3950   29.66 1.11e-09 ***
## Residuals   60  4.846  0.0808                     
## ---
## Signif. codes:  0 '***' 0.001 '**' 0.01 '*' 0.05 '.' 0.1 ' ' 1
\end{verbatim}

\begin{verbatim}
## F = 29.655 | F_critical = 3.15 | df1 = 2 | df2 = 60
\end{verbatim}

\begin{verbatim}
## Decision: Reject Ho  There are significant differences among the Rested groups.
\end{verbatim}

The ANOVA produces a highly significant F-statistic, well above the
critical value. This indicates that mean DASS scores differ
significantly among the three restfulness groups.

\subsubsection{Interpretation}\label{interpretation-2}

The statistical and visual evidence clearly indicate that restfulness is
strongly associated with mental-health symptoms. Students who report
being well-rested show substantially lower average DASS scores than
those who report being ``Somewhat'' or ``Not'' rested. The ANOVA
confirms that these differences are statistically meaningful.

In short, better sleep is linked with reduced symptoms of depression,
anxiety, and stress among university students. Although this does not
establish causation, the pattern is consistent and robust across the
analysis.

\newpage

\section{Research Question 4: Pre- vs Post-COVID Perceived Stress
(PSS)}\label{research-question-4-pre--vs-post-covid-perceived-stress-pss}

This analysis investigates whether students' perceived stress levels
changed significantly from \textbf{before} to \textbf{after} the
COVID-19 outbreak. The Perceived Stress Scale (PSS) contains 10 items
scored from 0--4, and higher totals indicate greater stress. Since the
same students completed both the pre- and post-COVID versions, a
\textbf{paired t-test} is appropriate to detect changes in stress
levels.

\begin{longtable}[]{@{}
  >{\raggedright\arraybackslash}p{(\columnwidth - 16\tabcolsep) * \real{0.1746}}
  >{\centering\arraybackslash}p{(\columnwidth - 16\tabcolsep) * \real{0.0952}}
  >{\centering\arraybackslash}p{(\columnwidth - 16\tabcolsep) * \real{0.0952}}
  >{\centering\arraybackslash}p{(\columnwidth - 16\tabcolsep) * \real{0.1270}}
  >{\centering\arraybackslash}p{(\columnwidth - 16\tabcolsep) * \real{0.1111}}
  >{\centering\arraybackslash}p{(\columnwidth - 16\tabcolsep) * \real{0.1270}}
  >{\centering\arraybackslash}p{(\columnwidth - 16\tabcolsep) * \real{0.0794}}
  >{\centering\arraybackslash}p{(\columnwidth - 16\tabcolsep) * \real{0.0794}}
  >{\centering\arraybackslash}p{(\columnwidth - 16\tabcolsep) * \real{0.1111}}@{}}
\caption{Descriptive Statistics for Pre- and Post-COVID PSS Total
Scores}\tabularnewline
\toprule\noalign{}
\begin{minipage}[b]{\linewidth}\raggedright
\end{minipage} & \begin{minipage}[b]{\linewidth}\centering
vars
\end{minipage} & \begin{minipage}[b]{\linewidth}\centering
n
\end{minipage} & \begin{minipage}[b]{\linewidth}\centering
mean
\end{minipage} & \begin{minipage}[b]{\linewidth}\centering
sd
\end{minipage} & \begin{minipage}[b]{\linewidth}\centering
median
\end{minipage} & \begin{minipage}[b]{\linewidth}\centering
min
\end{minipage} & \begin{minipage}[b]{\linewidth}\centering
max
\end{minipage} & \begin{minipage}[b]{\linewidth}\centering
range
\end{minipage} \\
\midrule\noalign{}
\endfirsthead
\toprule\noalign{}
\begin{minipage}[b]{\linewidth}\raggedright
\end{minipage} & \begin{minipage}[b]{\linewidth}\centering
vars
\end{minipage} & \begin{minipage}[b]{\linewidth}\centering
n
\end{minipage} & \begin{minipage}[b]{\linewidth}\centering
mean
\end{minipage} & \begin{minipage}[b]{\linewidth}\centering
sd
\end{minipage} & \begin{minipage}[b]{\linewidth}\centering
median
\end{minipage} & \begin{minipage}[b]{\linewidth}\centering
min
\end{minipage} & \begin{minipage}[b]{\linewidth}\centering
max
\end{minipage} & \begin{minipage}[b]{\linewidth}\centering
range
\end{minipage} \\
\midrule\noalign{}
\endhead
\bottomrule\noalign{}
\endlastfoot
pre\_total & 1 & 1193 & 22.506 & 6.792 & 23 & 0 & 42 & 42 \\
post\_total & 2 & 1193 & 25.376 & 6.985 & 26 & 0 & 42 & 42 \\
\end{longtable}

Students' post-COVID stress scores show a higher mean and increased
variability compared to pre-COVID scores, suggesting upward pressure on
stress levels after the onset of the pandemic.

\subsubsection{Boxplot Comparison}\label{boxplot-comparison}

\includegraphics{final_analysis_files/figure-latex/unnamed-chunk-19-1.pdf}
The boxplots show a clear upward shift in the distribution after
COVID-19: higher median, higher upper quartile, and more extreme values.

\subsubsection{Normality Check (QQ
Plots)}\label{normality-check-qq-plots}

\includegraphics{final_analysis_files/figure-latex/unnamed-chunk-20-1.pdf}
Both plots show mild deviations but are acceptable given the large
sample size (n = 1193), making the paired t-test robust.

\subsubsection{Paired t-Test}\label{paired-t-test}

\begin{verbatim}
## 
##  Paired t-test
## 
## data:  pss_data$pre_total and pss_data$post_total
## t = -15.13, df = 1192, p-value < 2.2e-16
## alternative hypothesis: true mean difference is not equal to 0
## 95 percent confidence interval:
##  -3.241302 -2.497173
## sample estimates:
## mean difference 
##       -2.869237
\end{verbatim}

\subsubsection{Effect Size (Cohen's d)}\label{effect-size-cohens-d}

\begin{verbatim}
## Cohen's d |       95% CI
## ------------------------
## 0.44      | [0.38, 0.50]
\end{verbatim}

Cohen's d is approximately 0.44, a medium effect size, meaning the
increase in stress is not only statistically significant but also
meaningful.

\subsubsection{Interpretation}\label{interpretation-3}

Students reported significantly higher perceived stress after the
COVID-19 outbreak. The paired t-test shows a large test statistic with p
\textless{} 0.001, and the effect size indicates a moderate increase in
stress, consistent across participants.

The boxplot and descriptive statistics reinforce this result: post-COVID
stress scores are shifted upward with greater variability.

Taken together, the evidence strongly supports that students experienced
increased psychological stress following the onset of COVID-19, both
statistically and practically significant.

\newpage

\section{Research Question 5: Hobbies and Emotional Regulation
(DERS)}\label{research-question-5-hobbies-and-emotional-regulation-ders}

This analysis investigates whether students' hobbies---both the amount
of time spent on them and the importance assigned to them---are
associated with difficulties in emotional regulation, as measured by the
DERS-16 scale. We screened all eight hobby-importance items, eight
hobby-time items, and two overall summary scores to identify which
variables showed meaningful relationships with emotional-regulation
difficulties. The goal was to determine whether certain leisure patterns
are linked to students' self-reported ability to manage emotions.

\subsubsection{Screening All Hobby
Variables}\label{screening-all-hobby-variables}

We began by examining the distribution and variability of all
hobby-related variables.

\begin{longtable}[]{@{}lrrr@{}}
\caption{Means and Standard Deviations for All Hobby
Variables}\tabularnewline
\toprule\noalign{}
Variable & Mean & SD & n \\
\midrule\noalign{}
\endfirsthead
\toprule\noalign{}
Variable & Mean & SD & n \\
\midrule\noalign{}
\endhead
\bottomrule\noalign{}
\endlastfoot
Hobbies\_Imp\_1 & 2.267 & 1.307 & 748 \\
Hobbies\_Imp\_2 & 2.116 & 1.091 & 748 \\
Hobbies\_Imp\_3 & 2.586 & 1.111 & 748 \\
Hobbies\_Imp\_4 & 3.229 & 1.017 & 748 \\
Hobbies\_Imp\_5 & 2.961 & 1.115 & 748 \\
Hobbies\_Imp\_6 & 4.167 & 0.798 & 748 \\
Hobbies\_Imp\_7 & 2.418 & 0.989 & 748 \\
Hobbies\_Imp\_8 & 3.270 & 1.063 & 748 \\
Hobbies\_Time\_1 & 1.717 & 1.131 & 748 \\
Hobbies\_Time\_2 & 1.509 & 0.925 & 748 \\
Hobbies\_Time\_3 & 2.028 & 1.293 & 748 \\
Hobbies\_Time\_4 & 4.162 & 1.388 & 748 \\
Hobbies\_Time\_5 & 1.999 & 1.336 & 748 \\
Hobbies\_Time\_6 & 5.352 & 1.501 & 748 \\
Hobbies\_Time\_7 & 1.488 & 0.840 & 748 \\
Hobbies\_Time\_8 & 2.485 & 1.370 & 748 \\
Imp\_overall & 2.877 & 0.468 & 748 \\
Time\_overall & 2.592 & 0.495 & 748 \\
\end{longtable}

Next, correlations were computed between each hobby variable and
DERS\_mean. This screening step identifies which items have the
strongest associations.

\begin{longtable}[]{@{}lrrrr@{}}
\caption{Correlations Between DERS\_mean and All Hobby Variables
(Screening Table)}\tabularnewline
\toprule\noalign{}
Variable & r & p\_value & n & abs\_r \\
\midrule\noalign{}
\endfirsthead
\toprule\noalign{}
Variable & r & p\_value & n & abs\_r \\
\midrule\noalign{}
\endhead
\bottomrule\noalign{}
\endlastfoot
Hobbies\_Time\_1 & -0.120 & 0.001 & 748 & 0.120 \\
Hobbies\_Imp\_4 & 0.108 & 0.003 & 748 & 0.108 \\
Hobbies\_Imp\_1 & -0.067 & 0.065 & 748 & 0.067 \\
Hobbies\_Imp\_8 & 0.067 & 0.066 & 748 & 0.067 \\
Hobbies\_Time\_4 & 0.063 & 0.084 & 748 & 0.063 \\
Imp\_overall & 0.050 & 0.174 & 748 & 0.050 \\
Hobbies\_Imp\_3 & 0.046 & 0.206 & 748 & 0.046 \\
Time\_overall & -0.034 & 0.357 & 748 & 0.034 \\
Hobbies\_Time\_2 & -0.027 & 0.466 & 748 & 0.027 \\
Hobbies\_Time\_5 & -0.025 & 0.488 & 748 & 0.025 \\
Hobbies\_Time\_6 & -0.021 & 0.564 & 748 & 0.021 \\
Hobbies\_Imp\_2 & 0.019 & 0.608 & 748 & 0.019 \\
Hobbies\_Imp\_6 & 0.017 & 0.643 & 748 & 0.017 \\
Hobbies\_Imp\_5 & 0.011 & 0.767 & 748 & 0.011 \\
Hobbies\_Time\_7 & 0.011 & 0.770 & 748 & 0.011 \\
Hobbies\_Time\_8 & -0.005 & 0.884 & 748 & 0.005 \\
Hobbies\_Imp\_7 & -0.005 & 0.896 & 748 & 0.005 \\
Hobbies\_Time\_3 & 0.003 & 0.944 & 748 & 0.003 \\
\end{longtable}

Variables considered ``noteworthy'' met both criteria:

\begin{itemize}
\tightlist
\item
  \textbar r\textbar{} ≥ 0.10
\item
  p \textless{} 0.05
\end{itemize}

\begin{longtable}[]{@{}lrrrr@{}}
\caption{Hobby Variables With Significant Correlation With
DERS\_mean}\tabularnewline
\toprule\noalign{}
Variable & r & p\_value & n & abs\_r \\
\midrule\noalign{}
\endfirsthead
\toprule\noalign{}
Variable & r & p\_value & n & abs\_r \\
\midrule\noalign{}
\endhead
\bottomrule\noalign{}
\endlastfoot
Hobbies\_Time\_1 & -0.120 & 0.001 & 748 & 0.120 \\
Hobbies\_Imp\_4 & 0.108 & 0.003 & 748 & 0.108 \\
\end{longtable}

Two items clearly stood out:

\begin{itemize}
\tightlist
\item
  Hobbies\_Time\_1: Hours spent on athletics / varsity / intramurals
\item
  Hobbies\_Imp\_4: Importance of watching online recreational content
  (Netflix, YouTube)
\end{itemize}

\subsubsection{Correlation Tests}\label{correlation-tests}

\begin{verbatim}
## 
##  Pearson's product-moment correlation
## 
## data:  DERS_mean and Hobbies_Time_1
## t = -3.3048, df = 746, p-value = 0.0009959
## alternative hypothesis: true correlation is not equal to 0
## 95 percent confidence interval:
##  -0.1901676 -0.0488571
## sample estimates:
##        cor 
## -0.1201208
\end{verbatim}

\begin{verbatim}
## 
##  Pearson's product-moment correlation
## 
## data:  DERS_mean and Hobbies_Imp_4
## t = 2.969, df = 746, p-value = 0.003083
## alternative hypothesis: true correlation is not equal to 0
## 95 percent confidence interval:
##  0.03666589 0.17836881
## sample estimates:
##       cor 
## 0.1080662
\end{verbatim}

\begin{itemize}
\tightlist
\item
  Hobbies\_Time\_1: small negative correlation
\item
  Hobbies\_Imp\_4: small positive correlation
\end{itemize}

\subsubsection{Linear Models}\label{linear-models}

Both linear models below confirm that, although the effect size was
small, they are statistically detectable due to the large sample size.

\begin{verbatim}
## 
## Call:
## lm(formula = DERS_mean ~ Hobbies_Time_1, data = Data_analysis)
## 
## Residuals:
##     Min      1Q  Median      3Q     Max 
## -1.9011 -0.7136 -0.0261  0.6964  2.1840 
## 
## Coefficients:
##                Estimate Std. Error t value Pr(>|t|)    
## (Intercept)     3.06112    0.06065  50.471  < 2e-16 ***
## Hobbies_Time_1 -0.09752    0.02951  -3.305 0.000996 ***
## ---
## Signif. codes:  0 '***' 0.001 '**' 0.01 '*' 0.05 '.' 0.1 ' ' 1
## 
## Residual standard error: 0.9123 on 746 degrees of freedom
## Multiple R-squared:  0.01443,    Adjusted R-squared:  0.01311 
## F-statistic: 10.92 on 1 and 746 DF,  p-value: 0.0009959
\end{verbatim}

\begin{verbatim}
## Analysis of Variance Table
## 
## Response: DERS_mean
##                 Df Sum Sq Mean Sq F value    Pr(>F)    
## Hobbies_Time_1   1   9.09  9.0907  10.922 0.0009959 ***
## Residuals      746 620.94  0.8324                      
## ---
## Signif. codes:  0 '***' 0.001 '**' 0.01 '*' 0.05 '.' 0.1 ' ' 1
\end{verbatim}

\begin{verbatim}
## 
## Call:
## lm(formula = DERS_mean ~ Hobbies_Imp_4, data = Data_analysis)
## 
## Residuals:
##      Min       1Q   Median       3Q      Max 
## -2.00416 -0.71903 -0.03153  0.69110  2.22623 
## 
## Coefficients:
##               Estimate Std. Error t value Pr(>|t|)    
## (Intercept)    2.57850    0.11130  23.167  < 2e-16 ***
## Hobbies_Imp_4  0.09763    0.03288   2.969  0.00308 ** 
## ---
## Signif. codes:  0 '***' 0.001 '**' 0.01 '*' 0.05 '.' 0.1 ' ' 1
## 
## Residual standard error: 0.9136 on 746 degrees of freedom
## Multiple R-squared:  0.01168,    Adjusted R-squared:  0.01035 
## F-statistic: 8.815 on 1 and 746 DF,  p-value: 0.003083
\end{verbatim}

\begin{verbatim}
## Analysis of Variance Table
## 
## Response: DERS_mean
##                Df Sum Sq Mean Sq F value   Pr(>F)   
## Hobbies_Imp_4   1   7.36  7.3576   8.815 0.003083 **
## Residuals     746 622.67  0.8347                    
## ---
## Signif. codes:  0 '***' 0.001 '**' 0.01 '*' 0.05 '.' 0.1 ' ' 1
\end{verbatim}

\subsubsection{Visualizations}\label{visualizations}

\includegraphics{final_analysis_files/figure-latex/unnamed-chunk-29-1.pdf}
\includegraphics{final_analysis_files/figure-latex/unnamed-chunk-29-2.pdf}

\subsubsection{Interpretation}\label{interpretation-4}

The screening process showed that most hobby variables had very weak or
negligible relationships with emotional-regulation difficulties.
However, two stood out with small yet statistically significant
associations:

\begin{enumerate}
\def\labelenumi{\arabic{enumi}.}
\tightlist
\item
  Hours spent in athletic activities (Hobbies\_Time\_1)
\end{enumerate}

Exhibits a negative relationship with DERS\_mean.

Students who spend more time in athletic or varsity activities tend to
report fewer difficulties regulating emotions.

This aligns with research showing that physical activity supports mood
regulation and stress recovery.

\begin{enumerate}
\def\labelenumi{\arabic{enumi}.}
\setcounter{enumi}{1}
\tightlist
\item
  Importance of recreational online content (Hobbies\_Imp\_4)
\end{enumerate}

Shows a positive association with DERS\_mean.

Students who rate online streaming (YouTube/Netflix) as more important
tend to score slightly higher on emotional-regulation difficulties.

While correlational, this could indicate coping through passive digital
consumption rather than active stress-management strategies.

Overall, hobby-related variables show subtle but meaningful links with
emotional regulation, suggesting that leisure choices---especially
physical activity and media-consumption habits---may play a modest role
in students' emotional well-being.

\newpage

\section{Conclusion and Future
Direction}\label{conclusion-and-future-direction}

Across all four research questions, the results point to a consistent
pattern: external factors such as program, sex, or international status
did not strongly relate to mental-health outcomes, while internal
factors---restfulness and emotional-regulation ability---showed clearer
associations.

Students in different programs showed only small differences in
emotion-regulation (DERS) scores. Seeking social support (SSS) also did
not vary meaningfully by program, sex, or international status. However,
students with higher emotional-regulation difficulties tended to seek
less social support, suggesting that personal emotional functioning
plays a bigger role than demographic characteristics. Feeling
well-rested showed a clear relationship with mental health: students who
slept well reported noticeably lower depression, anxiety, and stress
(DASS-21). Lastly, perceived stress (PSS) increased significantly after
COVID-19, with a medium effect size, indicating a real and meaningful
shift in student stress levels.

These findings should be viewed with some caution. The data are
self-reported, which can introduce bias, and the sample is not randomly
selected, since students participated voluntarily online. The dataset is
also cross-sectional, meaning we cannot establish cause and effect.
Scale scoring was followed according to provided guidelines, but
subjective interpretation of questions may still vary across
respondents.

Future studies could use a longitudinal design to track students over
time or include additional variables such as sleep duration, academic
workload, or social connectedness. More representative sampling across
institutions would also strengthen generalizability. Finally, combining
quantitative responses with qualitative insights could help explain why
students differ in stress, restfulness, or support-seeking behaviors.

Overall, the analysis suggests that personal well-being
factors---especially sleep and emotional regulation---play a larger role
in student mental health than demographic differences, and these areas
may be the most valuable targets for future support initiatives.

\newpage

\section{Appendix}\label{appendix}

\subsection{Research Question 1}\label{research-question-1}

\subsubsection{Install and Load
Libraries}\label{install-and-load-libraries}

\begin{Shaded}
\begin{Highlighting}[]
\FunctionTok{options}\NormalTok{(}\AttributeTok{repos =} \FunctionTok{c}\NormalTok{(}\AttributeTok{CRAN =} \StringTok{"https://cloud.r{-}project.org"}\NormalTok{))}
\FunctionTok{install.packages}\NormalTok{(}\FunctionTok{c}\NormalTok{(}\StringTok{"survey"}\NormalTok{))}
\end{Highlighting}
\end{Shaded}

\begin{verbatim}
## package 'survey' successfully unpacked and MD5 sums checked
## 
## The downloaded binary packages are in
##  C:\Users\toadcode\AppData\Local\Temp\RtmpW20DUw\downloaded_packages
\end{verbatim}

\subsubsection{Import Data}\label{import-data}

\begin{Shaded}
\begin{Highlighting}[]
\NormalTok{Data }\OtherTok{\textless{}{-}} \FunctionTok{read.csv}\NormalTok{(}\StringTok{"student\_mental\_health.csv"}\NormalTok{)[}\DecValTok{1}\SpecialCharTok{:}\DecValTok{1193}\NormalTok{, ]}
\FunctionTok{attach}\NormalTok{(Data)}
\end{Highlighting}
\end{Shaded}

\pagebreak

\subsubsection{Filter Data}\label{filter-data}

\begin{Shaded}
\begin{Highlighting}[]
\CommentTok{\# Remove observations where participants failed the catch question}
\NormalTok{passed\_catch }\OtherTok{=} \FunctionTok{filter}\NormalTok{(Data, Data}\SpecialCharTok{$}\NormalTok{Catch\_question }\SpecialCharTok{!=} \StringTok{"NA"}\NormalTok{)}

\CommentTok{\# Isolate the appropriate columns}
\NormalTok{this\_subset }\OtherTok{\textless{}{-}}\NormalTok{ passed\_catch[, }\FunctionTok{c}\NormalTok{(}\StringTok{"Program"}\NormalTok{, }\StringTok{"DERS\_1"}\NormalTok{, }\StringTok{"DERS\_2"}\NormalTok{, }\StringTok{"DERS\_3"}\NormalTok{, }\StringTok{"DERS\_4"}\NormalTok{, }
                                   \StringTok{"DERS\_5"}\NormalTok{, }\StringTok{"DERS\_6"}\NormalTok{, }\StringTok{"DERS\_7"}\NormalTok{, }\StringTok{"DERS\_8"}\NormalTok{, }
                                   \StringTok{"DERS\_9"}\NormalTok{, }\StringTok{"DERS\_10"}\NormalTok{, }\StringTok{"DERS\_11"}\NormalTok{, }\StringTok{"DERS\_12"}\NormalTok{, }
                                   \StringTok{"DERS\_13"}\NormalTok{, }\StringTok{"DERS\_14"}\NormalTok{, }\StringTok{"DERS\_15"}\NormalTok{,}\StringTok{"DERS\_16"}\NormalTok{)]}
                                   
\CommentTok{\# Add new column to data}
\NormalTok{this\_subset[}\StringTok{\textquotesingle{}DERS\_mean\textquotesingle{}}\NormalTok{] }\OtherTok{\textless{}{-}} \DecValTok{0}

\ControlFlowTok{for}\NormalTok{ (i }\ControlFlowTok{in} \DecValTok{1}\SpecialCharTok{:}\FunctionTok{length}\NormalTok{(this\_subset}\SpecialCharTok{$}\NormalTok{Program))\{}
  \CommentTok{\# Populate the new columns with the mean of DERS responses from each row}
\NormalTok{  this\_subset}\SpecialCharTok{$}\NormalTok{DERS\_mean[i] }\OtherTok{=} \FunctionTok{rowMeans}\NormalTok{(this\_subset[i, }\FunctionTok{c}\NormalTok{(}\DecValTok{2}\SpecialCharTok{:}\DecValTok{17}\NormalTok{)])}
  
\NormalTok{\}}
\end{Highlighting}
\end{Shaded}

\subsubsection{Boxplots of the Mean Response to DERS Questions Per
Program}\label{boxplots-of-the-mean-response-to-ders-questions-per-program}

\begin{Shaded}
\begin{Highlighting}[]
\NormalTok{this\_subset}\SpecialCharTok{$}\NormalTok{Program }\OtherTok{\textless{}{-}} \FunctionTok{as.factor}\NormalTok{(this\_subset}\SpecialCharTok{$}\NormalTok{Program)}

\FunctionTok{ggplot}\NormalTok{(this\_subset, }
       \FunctionTok{aes}\NormalTok{(}\AttributeTok{x=}\NormalTok{this\_subset}\SpecialCharTok{$}\NormalTok{Program, }
           \AttributeTok{y=}\NormalTok{this\_subset}\SpecialCharTok{$}\NormalTok{DERS\_mean, }
           \AttributeTok{fill=}\NormalTok{this\_subset}\SpecialCharTok{$}\NormalTok{Program)) }\SpecialCharTok{+}

\CommentTok{\# Label the scale}
\FunctionTok{scale\_fill\_discrete}\NormalTok{(}\StringTok{\textquotesingle{}Programs\textquotesingle{}}\NormalTok{, }
                    \AttributeTok{labels=}\FunctionTok{c}\NormalTok{(}\StringTok{\textquotesingle{}Business\textquotesingle{}}\NormalTok{, }\StringTok{\textquotesingle{}Education\textquotesingle{}}\NormalTok{, }\StringTok{\textquotesingle{}Engineering\textquotesingle{}}\NormalTok{, }
                    \StringTok{\textquotesingle{}Fine Arts\textquotesingle{}}\NormalTok{, }\StringTok{\textquotesingle{}Human \& Social Development\textquotesingle{}}\NormalTok{, }
                    \StringTok{\textquotesingle{}Humanities\textquotesingle{}}\NormalTok{, }\StringTok{\textquotesingle{}Sciences\textquotesingle{}}\NormalTok{, }
                    \StringTok{\textquotesingle{}Social Sciences\textquotesingle{}}\NormalTok{)) }\SpecialCharTok{+}

\FunctionTok{geom\_boxplot}\NormalTok{() }\SpecialCharTok{+}

\FunctionTok{labs}\NormalTok{(}\AttributeTok{title =} \StringTok{"Distribution of Average Response Per Program"}\NormalTok{,}
           \AttributeTok{x =} \StringTok{"Programs of Study"}\NormalTok{,}
           \AttributeTok{y =} \StringTok{"Response"}\NormalTok{) }\SpecialCharTok{+}

\FunctionTok{theme\_minimal}\NormalTok{()}\SpecialCharTok{+} 

\FunctionTok{theme}\NormalTok{(}\AttributeTok{plot.title =} \FunctionTok{element\_text}\NormalTok{(}\AttributeTok{hjust =} \FloatTok{0.5}\NormalTok{), }\CommentTok{\# center the title}
      \AttributeTok{axis.text.x =} \FunctionTok{element\_blank}\NormalTok{())}
\end{Highlighting}
\end{Shaded}

\includegraphics{final_analysis_files/figure-latex/unnamed-chunk-34-1.pdf}

\subsubsection{AoV and F Test}\label{aov-and-f-test}

\begin{Shaded}
\begin{Highlighting}[]
\NormalTok{results }\OtherTok{\textless{}{-}} \FunctionTok{aov}\NormalTok{(this\_subset}\SpecialCharTok{$}\NormalTok{DERS\_mean }\SpecialCharTok{\textasciitilde{}}\NormalTok{ this\_subset}\SpecialCharTok{$}\NormalTok{Program, }\AttributeTok{data =}\NormalTok{ this\_subset)}

\NormalTok{aov\_summary }\OtherTok{\textless{}{-}} \FunctionTok{summary}\NormalTok{(results)}

\NormalTok{aov\_table }\OtherTok{\textless{}{-}}\NormalTok{ aov\_summary[[}\DecValTok{1}\NormalTok{]]}

\CommentTok{\# F Test}
\NormalTok{f\_value }\OtherTok{\textless{}{-}}\NormalTok{ aov\_table[[}\StringTok{"F value"}\NormalTok{]][}\DecValTok{1}\NormalTok{]}

\NormalTok{df\_1 }\OtherTok{\textless{}{-}}\NormalTok{ aov\_table[[}\StringTok{"Df"}\NormalTok{]][}\DecValTok{1}\NormalTok{] }\CommentTok{\# degrees of freedom}
\NormalTok{df\_2 }\OtherTok{\textless{}{-}}\NormalTok{ aov\_table[[}\StringTok{"Df"}\NormalTok{]][}\DecValTok{2}\NormalTok{] }\CommentTok{\# degrees of freedom}

\NormalTok{f\_crit }\OtherTok{\textless{}{-}} \FunctionTok{qf}\NormalTok{(}\FloatTok{0.95}\NormalTok{, df\_1, df\_2)}

\ControlFlowTok{if}\NormalTok{ (f\_value }\SpecialCharTok{\textgreater{}}\NormalTok{ f\_crit) \{}
  \FunctionTok{print}\NormalTok{(}\StringTok{"Reject the null hypothesis."}\NormalTok{)}
  
\NormalTok{\} }\ControlFlowTok{else}\NormalTok{ \{}
  \FunctionTok{print}\NormalTok{(}\StringTok{"Fail to reject the null hypothesis."}\NormalTok{)}
  
\NormalTok{\}}
\end{Highlighting}
\end{Shaded}

\begin{verbatim}
## [1] "Reject the null hypothesis."
\end{verbatim}

\subsubsection{Condifence Intervals}\label{condifence-intervals}

\begin{Shaded}
\begin{Highlighting}[]
\NormalTok{Program }\OtherTok{\textless{}{-}} \DecValTok{1}\SpecialCharTok{:}\DecValTok{8}
\NormalTok{Mean }\OtherTok{\textless{}{-}} \DecValTok{0}
\NormalTok{Lower\_Bound }\OtherTok{\textless{}{-}} \DecValTok{0}
\NormalTok{Upper\_Bound }\OtherTok{\textless{}{-}} \DecValTok{0}

\NormalTok{ci\_summary }\OtherTok{=} \FunctionTok{data.frame}\NormalTok{(Program, Mean, Lower\_Bound, Upper\_Bound)}


\ControlFlowTok{for}\NormalTok{ (i }\ControlFlowTok{in} \DecValTok{1}\SpecialCharTok{:}\DecValTok{8}\NormalTok{)\{}
\NormalTok{  this\_data }\OtherTok{\textless{}{-}} \FunctionTok{subset}\NormalTok{(this\_subset}\SpecialCharTok{$}\NormalTok{DERS\_mean, this\_subset}\SpecialCharTok{$}\NormalTok{Program }\SpecialCharTok{==}\NormalTok{ i)}
  
\NormalTok{  t\_result }\OtherTok{\textless{}{-}} \FunctionTok{t.test}\NormalTok{(this\_data)}
  
\NormalTok{  this\_ci }\OtherTok{\textless{}{-}}\NormalTok{ t\_result}\SpecialCharTok{$}\NormalTok{conf.int}
  
\NormalTok{  ci\_summary}\SpecialCharTok{$}\NormalTok{Mean[i] }\OtherTok{=}\NormalTok{ t\_result}\SpecialCharTok{$}\NormalTok{estimate}
\NormalTok{  ci\_summary}\SpecialCharTok{$}\NormalTok{Lower\_Bound[i] }\OtherTok{=}\NormalTok{ this\_ci[}\DecValTok{1}\NormalTok{]}
\NormalTok{  ci\_summary}\SpecialCharTok{$}\NormalTok{Upper\_Bound[i] }\OtherTok{=}\NormalTok{ this\_ci[}\DecValTok{2}\NormalTok{]}
\NormalTok{\}}


\NormalTok{means }\OtherTok{\textless{}{-}}\NormalTok{ ci\_summary}\SpecialCharTok{$}\NormalTok{Mean}
\NormalTok{lower }\OtherTok{\textless{}{-}}\NormalTok{ ci\_summary}\SpecialCharTok{$}\NormalTok{Lower\_Bound}
\NormalTok{upper }\OtherTok{\textless{}{-}}\NormalTok{ ci\_summary}\SpecialCharTok{$}\NormalTok{Upper\_Bound}

\NormalTok{labels }\OtherTok{\textless{}{-}} \FunctionTok{c}\NormalTok{(}\StringTok{\textquotesingle{}Business\textquotesingle{}}\NormalTok{, }\StringTok{\textquotesingle{}Education\textquotesingle{}}\NormalTok{, }\StringTok{\textquotesingle{}Engineering\textquotesingle{}}\NormalTok{, }
                    \StringTok{\textquotesingle{}Fine Arts\textquotesingle{}}\NormalTok{, }\StringTok{\textquotesingle{}Human \& Social Development\textquotesingle{}}\NormalTok{, }
                    \StringTok{\textquotesingle{}Humanities\textquotesingle{}}\NormalTok{, }\StringTok{\textquotesingle{}Sciences\textquotesingle{}}\NormalTok{, }
                    \StringTok{\textquotesingle{}Social Sciences\textquotesingle{}}\NormalTok{)}

\FunctionTok{ggplot}\NormalTok{(ci\_summary, }\FunctionTok{aes}\NormalTok{(labels, means)) }\SpecialCharTok{+}
\CommentTok{\# Label the scale}
\FunctionTok{scale\_fill\_discrete}\NormalTok{(}\StringTok{\textquotesingle{}Programs\textquotesingle{}}\NormalTok{, }
                    \AttributeTok{labels=}\FunctionTok{c}\NormalTok{(}\StringTok{\textquotesingle{}Business\textquotesingle{}}\NormalTok{, }\StringTok{\textquotesingle{}Education\textquotesingle{}}\NormalTok{, }\StringTok{\textquotesingle{}Engineering\textquotesingle{}}\NormalTok{, }
                    \StringTok{\textquotesingle{}Fine Arts\textquotesingle{}}\NormalTok{, }\StringTok{\textquotesingle{}Human \& Social Development\textquotesingle{}}\NormalTok{, }
                    \StringTok{\textquotesingle{}Humanities\textquotesingle{}}\NormalTok{, }\StringTok{\textquotesingle{}Sciences\textquotesingle{}}\NormalTok{, }
                    \StringTok{\textquotesingle{}Social Sciences\textquotesingle{}}\NormalTok{)) }\SpecialCharTok{+}

\FunctionTok{geom\_point}\NormalTok{() }\SpecialCharTok{+}
\FunctionTok{geom\_errorbar}\NormalTok{(}\FunctionTok{aes}\NormalTok{(}\AttributeTok{ymin =}\NormalTok{ lower, }\AttributeTok{ymax =}\NormalTok{ upper)) }\SpecialCharTok{+} 
\FunctionTok{labs}\NormalTok{(}\AttributeTok{title =} \StringTok{"Confidence Interval of Mean of DERS Responses"}\NormalTok{,}
       \AttributeTok{x =} \StringTok{"Programs"}\NormalTok{,}
       \AttributeTok{y =} \StringTok{"Average Response to DERS"}\NormalTok{) }\SpecialCharTok{+}

\FunctionTok{theme\_minimal}\NormalTok{()}\SpecialCharTok{+} 
\FunctionTok{theme}\NormalTok{(}\AttributeTok{plot.title =} \FunctionTok{element\_text}\NormalTok{(}\AttributeTok{hjust =} \FloatTok{0.5}\NormalTok{)) }\SpecialCharTok{+} \CommentTok{\# center the title}
\FunctionTok{coord\_flip}\NormalTok{() }\SpecialCharTok{+}
\FunctionTok{scale\_y\_continuous}\NormalTok{(}\AttributeTok{limits =} \FunctionTok{c}\NormalTok{(}\DecValTok{2}\NormalTok{, }\DecValTok{4}\NormalTok{), }\AttributeTok{breaks =} \FunctionTok{seq}\NormalTok{(}\DecValTok{2}\NormalTok{, }\DecValTok{4}\NormalTok{, }\AttributeTok{by =} \FloatTok{0.25}\NormalTok{))}
\end{Highlighting}
\end{Shaded}

\includegraphics{final_analysis_files/figure-latex/unnamed-chunk-36-1.pdf}

\subsection{Research Question 2}\label{research-question-2}

\begin{Shaded}
\begin{Highlighting}[]
\FunctionTok{boxplot}\NormalTok{(SSS}\SpecialCharTok{\textasciitilde{}}\NormalTok{Program,}\AttributeTok{data=}\NormalTok{survey.data,}\AttributeTok{cex.axis=}\FloatTok{0.75}\NormalTok{)}
\end{Highlighting}
\end{Shaded}

\includegraphics{final_analysis_files/figure-latex/unnamed-chunk-38-1.pdf}

\begin{Shaded}
\begin{Highlighting}[]
\FunctionTok{oneway.test}\NormalTok{(SSS}\SpecialCharTok{\textasciitilde{}}\NormalTok{Program,}\AttributeTok{data=}\NormalTok{survey.data, }\AttributeTok{var.equal =} \ConstantTok{FALSE}\NormalTok{)}
\end{Highlighting}
\end{Shaded}

\begin{verbatim}
## 
##  One-way analysis of means (not assuming equal variances)
## 
## data:  SSS and Program
## F = 1.5207, num df = 7.00, denom df = 102.57, p-value = 0.1685
\end{verbatim}

\begin{Shaded}
\begin{Highlighting}[]
\FunctionTok{boxplot}\NormalTok{(SSS}\SpecialCharTok{\textasciitilde{}}\NormalTok{Sex,}\AttributeTok{data=}\NormalTok{survey.data)}
\end{Highlighting}
\end{Shaded}

\includegraphics{final_analysis_files/figure-latex/unnamed-chunk-40-1.pdf}

\begin{Shaded}
\begin{Highlighting}[]
\FunctionTok{t.test}\NormalTok{(survey.data[survey.data}\SpecialCharTok{$}\NormalTok{Sex}\SpecialCharTok{==}\StringTok{"Female"}\NormalTok{,}\StringTok{"SSS"}\NormalTok{],survey.data[survey.data}\SpecialCharTok{$}\NormalTok{Sex}\SpecialCharTok{==}\StringTok{"Male"}\NormalTok{,}\StringTok{"SSS"}\NormalTok{], }\StringTok{"two.sided"}\NormalTok{, }\DecValTok{0}\NormalTok{)}
\end{Highlighting}
\end{Shaded}

\begin{verbatim}
## 
##  Welch Two Sample t-test
## 
## data:  survey.data[survey.data$Sex == "Female", "SSS"] and survey.data[survey.data$Sex == "Male", "SSS"]
## t = 1.968, df = 116.74, p-value = 0.05144
## alternative hypothesis: true difference in means is not equal to 0
## 95 percent confidence interval:
##  -0.001005742  0.317077317
## sample estimates:
## mean of x mean of y 
##  3.300509  3.142473
\end{verbatim}

\begin{Shaded}
\begin{Highlighting}[]
\FunctionTok{boxplot}\NormalTok{(SSS}\SpecialCharTok{\textasciitilde{}}\NormalTok{International,}\AttributeTok{data=}\NormalTok{survey.data)}
\end{Highlighting}
\end{Shaded}

\includegraphics{final_analysis_files/figure-latex/unnamed-chunk-42-1.pdf}

\begin{Shaded}
\begin{Highlighting}[]
\FunctionTok{boxplot}\NormalTok{(SSS}\SpecialCharTok{\textasciitilde{}}\NormalTok{DERS, }\AttributeTok{data=}\NormalTok{survey.data)}
\end{Highlighting}
\end{Shaded}

\includegraphics{final_analysis_files/figure-latex/unnamed-chunk-43-1.pdf}

\begin{Shaded}
\begin{Highlighting}[]
\FunctionTok{boxplot}\NormalTok{(SSS}\SpecialCharTok{\textasciitilde{}}\NormalTok{DERS}\SpecialCharTok{\textgreater{}}\DecValTok{64}\NormalTok{, }\AttributeTok{data=}\NormalTok{survey.data)}
\end{Highlighting}
\end{Shaded}

\includegraphics{final_analysis_files/figure-latex/unnamed-chunk-44-1.pdf}

\begin{Shaded}
\begin{Highlighting}[]
\FunctionTok{t.test}\NormalTok{(survey.data[survey.data}\SpecialCharTok{$}\NormalTok{DERS}\SpecialCharTok{\textless{}=}\DecValTok{64}\NormalTok{,}\StringTok{"SSS"}\NormalTok{],survey.data[survey.data}\SpecialCharTok{$}\NormalTok{DERS}\SpecialCharTok{\textgreater{}}\DecValTok{64}\NormalTok{,}\StringTok{"SSS"}\NormalTok{], }\StringTok{"two.sided"}\NormalTok{, }\DecValTok{0}\NormalTok{)}
\end{Highlighting}
\end{Shaded}

\begin{verbatim}
## 
##  Welch Two Sample t-test
## 
## data:  survey.data[survey.data$DERS <= 64, "SSS"] and survey.data[survey.data$DERS > 64, "SSS"]
## t = 5.0255, df = 117.47, p-value = 1.816e-06
## alternative hypothesis: true difference in means is not equal to 0
## 95 percent confidence interval:
##  0.2425650 0.5580672
## sample estimates:
## mean of x mean of y 
##  3.331167  2.930851
\end{verbatim}

\subsection{Research Question 3}\label{research-question-3}

\begin{Shaded}
\begin{Highlighting}[]
\FunctionTok{load}\NormalTok{(}\StringTok{"Clean\_Data.RData"}\NormalTok{)}
\NormalTok{dat }\OtherTok{\textless{}{-}}\NormalTok{ y }\SpecialCharTok{\%\textgreater{}\%} \FunctionTok{mutate}\NormalTok{(}\AttributeTok{Rested =} \FunctionTok{as.numeric}\NormalTok{(Rested))}
\NormalTok{avg\_dass }\OtherTok{\textless{}{-}}\NormalTok{ dat }\SpecialCharTok{\%\textgreater{}\%}
  \FunctionTok{group\_by}\NormalTok{(Rested) }\SpecialCharTok{\%\textgreater{}\%}
  \FunctionTok{summarise}\NormalTok{(}
    \AttributeTok{n =} \FunctionTok{n}\NormalTok{(),}
    \FunctionTok{across}\NormalTok{(}\FunctionTok{starts\_with}\NormalTok{(}\StringTok{"DASS"}\NormalTok{), }\SpecialCharTok{\textasciitilde{}} \FunctionTok{mean}\NormalTok{(.x, }\AttributeTok{na.rm =} \ConstantTok{TRUE}\NormalTok{)),}
    \AttributeTok{.groups =} \StringTok{"drop"}
\NormalTok{  ) }\SpecialCharTok{\%\textgreater{}\%}
  \FunctionTok{rowwise}\NormalTok{() }\SpecialCharTok{\%\textgreater{}\%}
  \FunctionTok{mutate}\NormalTok{(}\AttributeTok{Overall\_DASS\_Average =} \FunctionTok{mean}\NormalTok{(}\FunctionTok{c\_across}\NormalTok{(}\FunctionTok{starts\_with}\NormalTok{(}\StringTok{"DASS"}\NormalTok{)), }\AttributeTok{na.rm =} \ConstantTok{TRUE}\NormalTok{)) }\SpecialCharTok{\%\textgreater{}\%}
  \FunctionTok{ungroup}\NormalTok{()}

\NormalTok{n\_row }\OtherTok{\textless{}{-}}\NormalTok{ avg\_dass }\SpecialCharTok{\%\textgreater{}\%}
  \FunctionTok{select}\NormalTok{(Rested, n) }\SpecialCharTok{\%\textgreater{}\%}
  \FunctionTok{pivot\_wider}\NormalTok{(}\AttributeTok{names\_from =}\NormalTok{ Rested, }\AttributeTok{values\_from =}\NormalTok{ n, }\AttributeTok{names\_sort =} \ConstantTok{TRUE}\NormalTok{) }\SpecialCharTok{\%\textgreater{}\%}
  \FunctionTok{mutate}\NormalTok{(}\AttributeTok{DASS\_Variable =} \StringTok{"n"}\NormalTok{) }\SpecialCharTok{\%\textgreater{}\%}
  \FunctionTok{select}\NormalTok{(DASS\_Variable, }\FunctionTok{everything}\NormalTok{())}

\NormalTok{avg\_dass\_long }\OtherTok{\textless{}{-}}\NormalTok{ avg\_dass }\SpecialCharTok{\%\textgreater{}\%}
  \FunctionTok{select}\NormalTok{(}\SpecialCharTok{{-}}\NormalTok{n) }\SpecialCharTok{\%\textgreater{}\%}
  \FunctionTok{pivot\_longer}\NormalTok{(}\AttributeTok{cols =} \SpecialCharTok{{-}}\NormalTok{Rested, }\AttributeTok{names\_to =} \StringTok{"DASS\_Variable"}\NormalTok{, }\AttributeTok{values\_to =} \StringTok{"Mean"}\NormalTok{) }\SpecialCharTok{\%\textgreater{}\%}
  \FunctionTok{pivot\_wider}\NormalTok{(}\AttributeTok{names\_from =}\NormalTok{ Rested, }\AttributeTok{values\_from =}\NormalTok{ Mean, }\AttributeTok{names\_sort =} \ConstantTok{TRUE}\NormalTok{)}

\NormalTok{final\_table }\OtherTok{\textless{}{-}} \FunctionTok{bind\_rows}\NormalTok{(n\_row, avg\_dass\_long)}

\NormalTok{knitr}\SpecialCharTok{::}\FunctionTok{kable}\NormalTok{(final\_table, }\AttributeTok{caption =} \StringTok{"Average DASS Scores by Rested Group"}\NormalTok{)}
\end{Highlighting}
\end{Shaded}

\begin{longtable}[]{@{}lrrr@{}}
\caption{Average DASS Scores by Rested Group}\tabularnewline
\toprule\noalign{}
DASS\_Variable & 1 & 2 & 3 \\
\midrule\noalign{}
\endfirsthead
\toprule\noalign{}
DASS\_Variable & 1 & 2 & 3 \\
\midrule\noalign{}
\endhead
\bottomrule\noalign{}
\endlastfoot
n & 155.000000 & 387.000000 & 206.000000 \\
DASS\_1 & 1.974193 & 2.333333 & 2.757282 \\
DASS\_2 & 1.793548 & 1.956072 & 2.150485 \\
DASS\_3 & 1.625806 & 1.878553 & 2.378641 \\
DASS\_4 & 1.470968 & 1.728682 & 1.975728 \\
DASS\_5 & 2.380645 & 2.764858 & 3.310680 \\
DASS\_6 & 1.993548 & 2.121447 & 2.461165 \\
DASS\_7 & 1.574194 & 1.829457 & 2.082524 \\
DASS\_8 & 2.161290 & 2.307494 & 2.762136 \\
DASS\_9 & 1.896774 & 2.180879 & 2.587379 \\
DASS\_10 & 1.812903 & 2.147287 & 2.718447 \\
DASS\_11 & 2.077419 & 2.320413 & 2.742718 \\
DASS\_12 & 2.103226 & 2.529716 & 3.077670 \\
DASS\_13 & 2.070968 & 2.372093 & 2.936893 \\
DASS\_14 & 1.864516 & 1.989664 & 2.296117 \\
DASS\_15 & 1.741936 & 1.956072 & 2.514563 \\
DASS\_16 & 1.774193 & 2.033592 & 2.621359 \\
DASS\_17 & 1.787097 & 1.935401 & 2.480583 \\
DASS\_18 & 1.767742 & 1.852713 & 2.208738 \\
DASS\_19 & 1.580645 & 1.863049 & 2.189320 \\
DASS\_20 & 1.741936 & 1.860465 & 2.223301 \\
DASS\_21 & 1.632258 & 1.780362 & 2.330097 \\
Overall\_DASS\_Average & 1.848848 & 2.082933 & 2.514563 \\
\end{longtable}

\begin{Shaded}
\begin{Highlighting}[]
\NormalTok{dat }\OtherTok{\textless{}{-}} \FunctionTok{read\_xlsx}\NormalTok{(}\StringTok{"average\_DASS\_transposed\_with\_n.xlsx"}\NormalTok{)}


\NormalTok{dat\_clean }\OtherTok{\textless{}{-}}\NormalTok{ dat }\SpecialCharTok{\%\textgreater{}\%}
  \FunctionTok{filter}\NormalTok{(}\SpecialCharTok{!}\NormalTok{(DASS\_Variable }\SpecialCharTok{\%in\%} \FunctionTok{c}\NormalTok{(}\StringTok{"n"}\NormalTok{, }\StringTok{"Overall\_DASS\_Average"}\NormalTok{)))}

\CommentTok{\# Convert from wide → long format}
\NormalTok{dat\_long }\OtherTok{\textless{}{-}}\NormalTok{ dat\_clean }\SpecialCharTok{\%\textgreater{}\%}
  \FunctionTok{pivot\_longer}\NormalTok{(}
    \AttributeTok{cols =} \FunctionTok{c}\NormalTok{(}\StringTok{\textasciigrave{}}\AttributeTok{1}\StringTok{\textasciigrave{}}\NormalTok{, }\StringTok{\textasciigrave{}}\AttributeTok{2}\StringTok{\textasciigrave{}}\NormalTok{, }\StringTok{\textasciigrave{}}\AttributeTok{3}\StringTok{\textasciigrave{}}\NormalTok{),}
    \AttributeTok{names\_to =} \StringTok{"Rested"}\NormalTok{,}
    \AttributeTok{values\_to =} \StringTok{"Mean\_DASS"}
\NormalTok{  ) }\SpecialCharTok{\%\textgreater{}\%}
  \FunctionTok{mutate}\NormalTok{(}
    \AttributeTok{Rested =} \FunctionTok{recode}\NormalTok{(Rested,}
                    \StringTok{"1"} \OtherTok{=} \StringTok{"Yes"}\NormalTok{,}
                    \StringTok{"2"} \OtherTok{=} \StringTok{"Somewhat"}\NormalTok{,}
                    \StringTok{"3"} \OtherTok{=} \StringTok{"No"}\NormalTok{),}
    \AttributeTok{Rested =} \FunctionTok{factor}\NormalTok{(Rested, }\AttributeTok{levels =} \FunctionTok{c}\NormalTok{(}\StringTok{"Yes"}\NormalTok{, }\StringTok{"Somewhat"}\NormalTok{, }\StringTok{"No"}\NormalTok{))}
\NormalTok{  )}

\FunctionTok{ggplot}\NormalTok{(dat\_long, }\FunctionTok{aes}\NormalTok{(}\AttributeTok{x =}\NormalTok{ Rested, }\AttributeTok{y =}\NormalTok{ Mean\_DASS, }\AttributeTok{fill =}\NormalTok{ Rested)) }\SpecialCharTok{+}
  \FunctionTok{geom\_boxplot}\NormalTok{(}\AttributeTok{width =} \FloatTok{0.6}\NormalTok{, }\AttributeTok{outlier.alpha =} \FloatTok{0.3}\NormalTok{) }\SpecialCharTok{+}
  \CommentTok{\# (Optional) add jitter points for each DASS item}
  \CommentTok{\# geom\_jitter(width = 0.15, alpha = 0.6, size = 1.8) +}
  \FunctionTok{scale\_fill\_manual}\NormalTok{(}\AttributeTok{values =} \FunctionTok{c}\NormalTok{(}\StringTok{"\#99ff99"}\NormalTok{, }\StringTok{"\#99ccff"}\NormalTok{, }\StringTok{"\#ff9999"}\NormalTok{)) }\SpecialCharTok{+}
  \FunctionTok{labs}\NormalTok{(}
    \AttributeTok{title =} \StringTok{"Average DASS Results by Rested"}\NormalTok{,}
    \AttributeTok{x =} \StringTok{"Rested Response"}\NormalTok{,}
    \AttributeTok{y =} \StringTok{"Average DASS Value"}\NormalTok{,}
    \AttributeTok{fill =} \StringTok{"Rested"}
\NormalTok{  ) }\SpecialCharTok{+}
  \FunctionTok{theme\_minimal}\NormalTok{(}\AttributeTok{base\_size =} \DecValTok{12}\NormalTok{) }\SpecialCharTok{+}
  \FunctionTok{theme}\NormalTok{(}
    \AttributeTok{plot.title =} \FunctionTok{element\_text}\NormalTok{(}\AttributeTok{hjust =} \FloatTok{0.5}\NormalTok{, }\AttributeTok{face =} \StringTok{"bold"}\NormalTok{),}
    \AttributeTok{legend.position =} \StringTok{"none"}
\NormalTok{  )}
\end{Highlighting}
\end{Shaded}

\includegraphics{final_analysis_files/figure-latex/unnamed-chunk-48-1.pdf}

\begin{Shaded}
\begin{Highlighting}[]
\NormalTok{dat }\OtherTok{\textless{}{-}} \FunctionTok{read\_xlsx}\NormalTok{(}\StringTok{"average\_DASS\_transposed\_with\_n.xlsx"}\NormalTok{)}


\NormalTok{dat\_clean }\OtherTok{\textless{}{-}}\NormalTok{ dat }\SpecialCharTok{\%\textgreater{}\%}
  \FunctionTok{filter}\NormalTok{(}\SpecialCharTok{!}\NormalTok{(DASS\_Variable }\SpecialCharTok{\%in\%} \FunctionTok{c}\NormalTok{(}\StringTok{"n"}\NormalTok{, }\StringTok{"Overall\_DASS\_Average"}\NormalTok{)))}

\NormalTok{dat\_long }\OtherTok{\textless{}{-}}\NormalTok{ dat\_clean }\SpecialCharTok{\%\textgreater{}\%}
  \FunctionTok{pivot\_longer}\NormalTok{(}
    \AttributeTok{cols =} \FunctionTok{c}\NormalTok{(}\StringTok{\textasciigrave{}}\AttributeTok{1}\StringTok{\textasciigrave{}}\NormalTok{, }\StringTok{\textasciigrave{}}\AttributeTok{2}\StringTok{\textasciigrave{}}\NormalTok{, }\StringTok{\textasciigrave{}}\AttributeTok{3}\StringTok{\textasciigrave{}}\NormalTok{),}
    \AttributeTok{names\_to =} \StringTok{"Rested"}\NormalTok{,}
    \AttributeTok{values\_to =} \StringTok{"Mean\_DASS"}
\NormalTok{  ) }\SpecialCharTok{\%\textgreater{}\%}
  \FunctionTok{mutate}\NormalTok{(}
    \AttributeTok{Rested =} \FunctionTok{factor}\NormalTok{(Rested,}
                    \AttributeTok{levels =} \FunctionTok{c}\NormalTok{(}\StringTok{"1"}\NormalTok{, }\StringTok{"2"}\NormalTok{, }\StringTok{"3"}\NormalTok{),}
                    \AttributeTok{labels =} \FunctionTok{c}\NormalTok{(}\StringTok{"Yes"}\NormalTok{, }\StringTok{"Somewhat"}\NormalTok{, }\StringTok{"No"}\NormalTok{))}
\NormalTok{  )}


\NormalTok{anova\_model }\OtherTok{\textless{}{-}} \FunctionTok{aov}\NormalTok{(Mean\_DASS }\SpecialCharTok{\textasciitilde{}}\NormalTok{ Rested, }\AttributeTok{data =}\NormalTok{ dat\_long)}
\NormalTok{anova\_summary }\OtherTok{\textless{}{-}} \FunctionTok{summary}\NormalTok{(anova\_model)}
\NormalTok{anova\_summary}
\end{Highlighting}
\end{Shaded}

\begin{verbatim}
##             Df Sum Sq Mean Sq F value   Pr(>F)    
## Rested       2  4.790  2.3950   29.66 1.11e-09 ***
## Residuals   60  4.846  0.0808                     
## ---
## Signif. codes:  0 '***' 0.001 '**' 0.01 '*' 0.05 '.' 0.1 ' ' 1
\end{verbatim}

\begin{Shaded}
\begin{Highlighting}[]
\NormalTok{anova\_table }\OtherTok{\textless{}{-}}\NormalTok{ anova\_summary[[}\DecValTok{1}\NormalTok{]]}
\NormalTok{F\_value }\OtherTok{\textless{}{-}}\NormalTok{ anova\_table[[}\StringTok{"F value"}\NormalTok{]][}\DecValTok{1}\NormalTok{]}
\NormalTok{df1 }\OtherTok{\textless{}{-}}\NormalTok{ anova\_table[[}\StringTok{"Df"}\NormalTok{]][}\DecValTok{1}\NormalTok{]     }\CommentTok{\# Between groups}
\NormalTok{df2 }\OtherTok{\textless{}{-}}\NormalTok{ anova\_table[[}\StringTok{"Df"}\NormalTok{]][}\DecValTok{2}\NormalTok{]     }\CommentTok{\# Within groups}
\NormalTok{alpha }\OtherTok{\textless{}{-}} \FloatTok{0.05}

\NormalTok{F\_critical }\OtherTok{\textless{}{-}} \FunctionTok{qf}\NormalTok{(}\DecValTok{1} \SpecialCharTok{{-}}\NormalTok{ alpha, df1, df2)}

\FunctionTok{cat}\NormalTok{(}\StringTok{"F ="}\NormalTok{, }\FunctionTok{round}\NormalTok{(F\_value, }\DecValTok{3}\NormalTok{),}
    \StringTok{"| F\_critical ="}\NormalTok{, }\FunctionTok{round}\NormalTok{(F\_critical, }\DecValTok{3}\NormalTok{),}
    \StringTok{"| df1 ="}\NormalTok{, df1, }\StringTok{"| df2 ="}\NormalTok{, df2, }\StringTok{"}\SpecialCharTok{\textbackslash{}n}\StringTok{"}\NormalTok{)}
\end{Highlighting}
\end{Shaded}

\begin{verbatim}
## F = 29.655 | F_critical = 3.15 | df1 = 2 | df2 = 60
\end{verbatim}

\begin{Shaded}
\begin{Highlighting}[]
\ControlFlowTok{if}\NormalTok{ (F\_value }\SpecialCharTok{\textgreater{}}\NormalTok{ F\_critical) \{}
  \FunctionTok{cat}\NormalTok{(}\StringTok{"Decision: Reject Ho  There are significant differences among the Rested groups.}\SpecialCharTok{\textbackslash{}n}\StringTok{"}\NormalTok{)}
\NormalTok{\} }\ControlFlowTok{else}\NormalTok{ \{}
  \FunctionTok{cat}\NormalTok{(}\StringTok{"Decision: Fail to reject Ho — No significant differences among the Rested groups.}\SpecialCharTok{\textbackslash{}n}\StringTok{"}\NormalTok{)}
\NormalTok{\}}
\end{Highlighting}
\end{Shaded}

\begin{verbatim}
## Decision: Reject Ho  There are significant differences among the Rested groups.
\end{verbatim}

\subsection{Research Question 4}\label{research-question-4}

\subsubsection{Install Sampling and Survey
Package}\label{install-sampling-and-survey-package}

\begin{Shaded}
\begin{Highlighting}[]
\FunctionTok{options}\NormalTok{(}\AttributeTok{repos =} \FunctionTok{c}\NormalTok{(}\AttributeTok{CRAN =} \StringTok{"https://cloud.r{-}project.org"}\NormalTok{))}
\FunctionTok{install.packages}\NormalTok{(}\FunctionTok{c}\NormalTok{(}\StringTok{"tidyverse"}\NormalTok{, }\StringTok{"psych"}\NormalTok{, }\StringTok{"effectsize"}\NormalTok{, }\StringTok{"knitr"}\NormalTok{))}
\end{Highlighting}
\end{Shaded}

\subsubsection{Load the Required
Libraries}\label{load-the-required-libraries}

\begin{Shaded}
\begin{Highlighting}[]
\FunctionTok{library}\NormalTok{(tidyverse)}
\FunctionTok{library}\NormalTok{(psych)}
\FunctionTok{library}\NormalTok{(effectsize)}
\FunctionTok{library}\NormalTok{(knitr)}
\end{Highlighting}
\end{Shaded}

\subsubsection{Import the Data File into
R}\label{import-the-data-file-into-r}

\begin{Shaded}
\begin{Highlighting}[]
\CommentTok{\# Read/Import the data file in R}
\NormalTok{student\_mental\_health }\OtherTok{\textless{}{-}} \FunctionTok{read.csv}\NormalTok{(}\StringTok{"student\_mental\_health.csv"}\NormalTok{)[}\DecValTok{1}\SpecialCharTok{:}\DecValTok{1193}\NormalTok{, ]}
\FunctionTok{attach}\NormalTok{(student\_mental\_health)}
\end{Highlighting}
\end{Shaded}

\subsubsection{Select \& Prepare PSS
Variables}\label{select-prepare-pss-variables}

\begin{Shaded}
\begin{Highlighting}[]
\CommentTok{\# Select only the relevant columns}
\NormalTok{pss\_vars\_pre  }\OtherTok{\textless{}{-}} \FunctionTok{paste0}\NormalTok{(}\StringTok{\textquotesingle{}Pre\_PSS\_\textquotesingle{}}\NormalTok{, }\DecValTok{1}\SpecialCharTok{:}\DecValTok{10}\NormalTok{)}
\NormalTok{pss\_vars\_post }\OtherTok{\textless{}{-}} \FunctionTok{paste0}\NormalTok{(}\StringTok{\textquotesingle{}Post\_PSS\_\textquotesingle{}}\NormalTok{, }\DecValTok{1}\SpecialCharTok{:}\DecValTok{10}\NormalTok{)}

\NormalTok{pss\_data }\OtherTok{\textless{}{-}}\NormalTok{ student\_mental\_health }\SpecialCharTok{\%\textgreater{}\%}
  \FunctionTok{select}\NormalTok{(}\FunctionTok{all\_of}\NormalTok{(}\FunctionTok{c}\NormalTok{(pss\_vars\_pre, pss\_vars\_post)))}
\end{Highlighting}
\end{Shaded}

\subsubsection{Reverse-Code Items 4, 5, 7,
8}\label{reverse-code-items-4-5-7-8}

\begin{Shaded}
\begin{Highlighting}[]
\NormalTok{reverse\_items }\OtherTok{\textless{}{-}} \FunctionTok{c}\NormalTok{(}\DecValTok{4}\NormalTok{, }\DecValTok{5}\NormalTok{, }\DecValTok{7}\NormalTok{, }\DecValTok{8}\NormalTok{)}

\ControlFlowTok{for}\NormalTok{ (i }\ControlFlowTok{in}\NormalTok{ reverse\_items) \{}
\NormalTok{    pss\_data[[}\FunctionTok{paste0}\NormalTok{(}\StringTok{\textquotesingle{}Pre\_PSS\_\textquotesingle{}}\NormalTok{, i)]] }\OtherTok{\textless{}{-}} \DecValTok{4} \SpecialCharTok{{-}}\NormalTok{ pss\_data[[}\FunctionTok{paste0}\NormalTok{(}\StringTok{\textquotesingle{}Pre\_PSS\_\textquotesingle{}}\NormalTok{, i)]]}
\NormalTok{    pss\_data[[}\FunctionTok{paste0}\NormalTok{(}\StringTok{\textquotesingle{}Post\_PSS\_\textquotesingle{}}\NormalTok{, i)]] }\OtherTok{\textless{}{-}} \DecValTok{4} \SpecialCharTok{{-}}\NormalTok{ pss\_data[[}\FunctionTok{paste0}\NormalTok{(}\StringTok{\textquotesingle{}Post\_PSS\_\textquotesingle{}}\NormalTok{, i)]]}
\NormalTok{\}}
\end{Highlighting}
\end{Shaded}

\subsubsection{Compute Total Scores}\label{compute-total-scores}

\begin{Shaded}
\begin{Highlighting}[]
\NormalTok{pss\_data }\OtherTok{\textless{}{-}}\NormalTok{ pss\_data }\SpecialCharTok{\%\textgreater{}\%}
\FunctionTok{mutate}\NormalTok{(}
    \AttributeTok{pre\_total =} \FunctionTok{rowSums}\NormalTok{(}\FunctionTok{select}\NormalTok{(., }\FunctionTok{all\_of}\NormalTok{(pss\_vars\_pre)), }\AttributeTok{na.rm =} \ConstantTok{TRUE}\NormalTok{),}
    \AttributeTok{post\_total =} \FunctionTok{rowSums}\NormalTok{(}\FunctionTok{select}\NormalTok{(., }\FunctionTok{all\_of}\NormalTok{(pss\_vars\_post)), }\AttributeTok{na.rm =} \ConstantTok{TRUE}\NormalTok{)}
\NormalTok{)}
\end{Highlighting}
\end{Shaded}

\subsubsection{Descriptive Statistics}\label{descriptive-statistics}

\begin{Shaded}
\begin{Highlighting}[]
\FunctionTok{describe}\NormalTok{(pss\_data[, }\FunctionTok{c}\NormalTok{(}\StringTok{\textquotesingle{}pre\_total\textquotesingle{}}\NormalTok{, }\StringTok{\textquotesingle{}post\_total\textquotesingle{}}\NormalTok{)])}
\end{Highlighting}
\end{Shaded}

\begin{verbatim}
##            vars    n  mean   sd median trimmed  mad min max range  skew
## pre_total     1 1193 22.51 6.79     23   22.50 5.93   0  42    42  0.01
## post_total    2 1193 25.38 6.98     26   25.44 7.41   0  42    42 -0.14
##            kurtosis  se
## pre_total     -0.03 0.2
## post_total     0.02 0.2
\end{verbatim}

\begin{Shaded}
\begin{Highlighting}[]
\CommentTok{\# Boxplot comparison (pre shown on the left)}
\NormalTok{pss\_data }\SpecialCharTok{\%\textgreater{}\%}
  \FunctionTok{pivot\_longer}\NormalTok{(}\AttributeTok{cols =} \FunctionTok{c}\NormalTok{(pre\_total, post\_total),}
               \AttributeTok{names\_to =} \StringTok{\textquotesingle{}Period\textquotesingle{}}\NormalTok{, }\AttributeTok{values\_to =} \StringTok{\textquotesingle{}Score\textquotesingle{}}\NormalTok{) }\SpecialCharTok{\%\textgreater{}\%}
  \FunctionTok{mutate}\NormalTok{(}\AttributeTok{Period =} \FunctionTok{factor}\NormalTok{(Period, }\AttributeTok{levels =} \FunctionTok{c}\NormalTok{(}\StringTok{\textquotesingle{}pre\_total\textquotesingle{}}\NormalTok{, }\StringTok{\textquotesingle{}post\_total\textquotesingle{}}\NormalTok{))) }\SpecialCharTok{\%\textgreater{}\%}  \CommentTok{\# enforce order}
  \FunctionTok{ggplot}\NormalTok{(}\FunctionTok{aes}\NormalTok{(}\AttributeTok{x =}\NormalTok{ Period, }\AttributeTok{y =}\NormalTok{ Score, }\AttributeTok{fill =}\NormalTok{ Period)) }\SpecialCharTok{+}
  \FunctionTok{geom\_boxplot}\NormalTok{() }\SpecialCharTok{+}
  \FunctionTok{labs}\NormalTok{(}
    \AttributeTok{title =} \StringTok{\textquotesingle{}Perceived Stress Levels Before vs After COVID{-}19\textquotesingle{}}\NormalTok{,}
    \AttributeTok{y =} \StringTok{\textquotesingle{}PSS Total Score\textquotesingle{}}\NormalTok{,}
    \AttributeTok{x =} \StringTok{\textquotesingle{}\textquotesingle{}}
\NormalTok{  ) }\SpecialCharTok{+}
  \FunctionTok{theme\_minimal}\NormalTok{()}
\end{Highlighting}
\end{Shaded}

\includegraphics{final_analysis_files/figure-latex/unnamed-chunk-55-1.pdf}

\begin{Shaded}
\begin{Highlighting}[]
\CommentTok{\# QQ plots for normality check}
\FunctionTok{par}\NormalTok{(}\AttributeTok{mfrow =} \FunctionTok{c}\NormalTok{(}\DecValTok{1}\NormalTok{, }\DecValTok{2}\NormalTok{))  }\CommentTok{\# two plots side by side}

\FunctionTok{qqnorm}\NormalTok{(pss\_data}\SpecialCharTok{$}\NormalTok{pre\_total, }\AttributeTok{main =} \StringTok{\textquotesingle{}QQ Plot {-} Pre{-}COVID PSS\textquotesingle{}}\NormalTok{)}
\FunctionTok{qqline}\NormalTok{(pss\_data}\SpecialCharTok{$}\NormalTok{pre\_total, }\AttributeTok{col =} \StringTok{\textquotesingle{}red\textquotesingle{}}\NormalTok{)}

\FunctionTok{qqnorm}\NormalTok{(pss\_data}\SpecialCharTok{$}\NormalTok{post\_total, }\AttributeTok{main =} \StringTok{\textquotesingle{}QQ Plot {-} Post{-}COVID PSS\textquotesingle{}}\NormalTok{)}
\FunctionTok{qqline}\NormalTok{(pss\_data}\SpecialCharTok{$}\NormalTok{post\_total, }\AttributeTok{col =} \StringTok{\textquotesingle{}red\textquotesingle{}}\NormalTok{)}
\end{Highlighting}
\end{Shaded}

\includegraphics{final_analysis_files/figure-latex/unnamed-chunk-55-2.pdf}

\begin{Shaded}
\begin{Highlighting}[]
\FunctionTok{par}\NormalTok{(}\AttributeTok{mfrow =} \FunctionTok{c}\NormalTok{(}\DecValTok{1}\NormalTok{, }\DecValTok{1}\NormalTok{))  }\CommentTok{\# reset layout}
\end{Highlighting}
\end{Shaded}

\subsubsection{Paired t-Test}\label{paired-t-test-1}

\begin{Shaded}
\begin{Highlighting}[]
\NormalTok{t\_test\_result }\OtherTok{\textless{}{-}} \FunctionTok{t.test}\NormalTok{(pss\_data}\SpecialCharTok{$}\NormalTok{pre\_total, pss\_data}\SpecialCharTok{$}\NormalTok{post\_total, }\AttributeTok{paired =} \ConstantTok{TRUE}\NormalTok{)}
\NormalTok{t\_test\_result}
\end{Highlighting}
\end{Shaded}

\begin{verbatim}
## 
##  Paired t-test
## 
## data:  pss_data$pre_total and pss_data$post_total
## t = -15.13, df = 1192, p-value < 2.2e-16
## alternative hypothesis: true mean difference is not equal to 0
## 95 percent confidence interval:
##  -3.241302 -2.497173
## sample estimates:
## mean difference 
##       -2.869237
\end{verbatim}

\subsubsection{Effect Size}\label{effect-size}

\begin{Shaded}
\begin{Highlighting}[]
\FunctionTok{cohens\_d}\NormalTok{(pss\_data}\SpecialCharTok{$}\NormalTok{post\_total, pss\_data}\SpecialCharTok{$}\NormalTok{pre\_total, }\AttributeTok{paired =} \ConstantTok{TRUE}\NormalTok{)}
\end{Highlighting}
\end{Shaded}

\begin{verbatim}
## Cohen's d |       95% CI
## ------------------------
## 0.44      | [0.38, 0.50]
\end{verbatim}

\subsubsection{Result}\label{result}

\begin{Shaded}
\begin{Highlighting}[]
\CommentTok{\# Compute descriptive statistics}
\NormalTok{desc\_stats }\OtherTok{\textless{}{-}} \FunctionTok{describe}\NormalTok{(pss\_data[, }\FunctionTok{c}\NormalTok{(}\StringTok{\textquotesingle{}pre\_total\textquotesingle{}}\NormalTok{, }\StringTok{\textquotesingle{}post\_total\textquotesingle{}}\NormalTok{)])}
\CommentTok{\# Round all numeric columns to 4 decimals}
\NormalTok{desc\_summary }\OtherTok{\textless{}{-}}\NormalTok{ desc\_stats }\SpecialCharTok{\%\textgreater{}\%}
  \FunctionTok{select}\NormalTok{(vars, n, mean, sd, median, min, max, range, skew, kurtosis) }\SpecialCharTok{\%\textgreater{}\%}
  \FunctionTok{mutate}\NormalTok{(}\FunctionTok{across}\NormalTok{(}\FunctionTok{where}\NormalTok{(is.numeric), }\SpecialCharTok{\textasciitilde{}} \FunctionTok{round}\NormalTok{(., }\DecValTok{4}\NormalTok{)))}

\CommentTok{\# Display formatted table}
\FunctionTok{kable}\NormalTok{(desc\_summary,}
      \AttributeTok{caption =} \StringTok{"Descriptive Statistics for Pre{-} and Post{-}COVID Perceived Stress Scores"}\NormalTok{,}
      \AttributeTok{align =} \StringTok{\textquotesingle{}c\textquotesingle{}}\NormalTok{)}
\end{Highlighting}
\end{Shaded}

\begin{longtable}[]{@{}
  >{\raggedright\arraybackslash}p{(\columnwidth - 20\tabcolsep) * \real{0.1310}}
  >{\centering\arraybackslash}p{(\columnwidth - 20\tabcolsep) * \real{0.0714}}
  >{\centering\arraybackslash}p{(\columnwidth - 20\tabcolsep) * \real{0.0714}}
  >{\centering\arraybackslash}p{(\columnwidth - 20\tabcolsep) * \real{0.1071}}
  >{\centering\arraybackslash}p{(\columnwidth - 20\tabcolsep) * \real{0.0952}}
  >{\centering\arraybackslash}p{(\columnwidth - 20\tabcolsep) * \real{0.0952}}
  >{\centering\arraybackslash}p{(\columnwidth - 20\tabcolsep) * \real{0.0595}}
  >{\centering\arraybackslash}p{(\columnwidth - 20\tabcolsep) * \real{0.0595}}
  >{\centering\arraybackslash}p{(\columnwidth - 20\tabcolsep) * \real{0.0833}}
  >{\centering\arraybackslash}p{(\columnwidth - 20\tabcolsep) * \real{0.1071}}
  >{\centering\arraybackslash}p{(\columnwidth - 20\tabcolsep) * \real{0.1190}}@{}}
\caption{Descriptive Statistics for Pre- and Post-COVID Perceived Stress
Scores}\tabularnewline
\toprule\noalign{}
\begin{minipage}[b]{\linewidth}\raggedright
\end{minipage} & \begin{minipage}[b]{\linewidth}\centering
vars
\end{minipage} & \begin{minipage}[b]{\linewidth}\centering
n
\end{minipage} & \begin{minipage}[b]{\linewidth}\centering
mean
\end{minipage} & \begin{minipage}[b]{\linewidth}\centering
sd
\end{minipage} & \begin{minipage}[b]{\linewidth}\centering
median
\end{minipage} & \begin{minipage}[b]{\linewidth}\centering
min
\end{minipage} & \begin{minipage}[b]{\linewidth}\centering
max
\end{minipage} & \begin{minipage}[b]{\linewidth}\centering
range
\end{minipage} & \begin{minipage}[b]{\linewidth}\centering
skew
\end{minipage} & \begin{minipage}[b]{\linewidth}\centering
kurtosis
\end{minipage} \\
\midrule\noalign{}
\endfirsthead
\toprule\noalign{}
\begin{minipage}[b]{\linewidth}\raggedright
\end{minipage} & \begin{minipage}[b]{\linewidth}\centering
vars
\end{minipage} & \begin{minipage}[b]{\linewidth}\centering
n
\end{minipage} & \begin{minipage}[b]{\linewidth}\centering
mean
\end{minipage} & \begin{minipage}[b]{\linewidth}\centering
sd
\end{minipage} & \begin{minipage}[b]{\linewidth}\centering
median
\end{minipage} & \begin{minipage}[b]{\linewidth}\centering
min
\end{minipage} & \begin{minipage}[b]{\linewidth}\centering
max
\end{minipage} & \begin{minipage}[b]{\linewidth}\centering
range
\end{minipage} & \begin{minipage}[b]{\linewidth}\centering
skew
\end{minipage} & \begin{minipage}[b]{\linewidth}\centering
kurtosis
\end{minipage} \\
\midrule\noalign{}
\endhead
\bottomrule\noalign{}
\endlastfoot
pre\_total & 1 & 1193 & 22.5063 & 6.7924 & 23 & 0 & 42 & 42 & 0.0148 &
-0.0308 \\
post\_total & 2 & 1193 & 25.3755 & 6.9850 & 26 & 0 & 42 & 42 & -0.1448 &
0.0243 \\
\end{longtable}

\subsection{Research Question 5}\label{research-question-5}

\begin{Shaded}
\begin{Highlighting}[]
\NormalTok{Data }\OtherTok{\textless{}{-}} \FunctionTok{read.csv}\NormalTok{(}\StringTok{"student\_mental\_health.csv"}\NormalTok{)[}\DecValTok{1}\SpecialCharTok{:}\DecValTok{1193}\NormalTok{, ]}

\NormalTok{ders\_items }\OtherTok{\textless{}{-}} \FunctionTok{paste0}\NormalTok{(}\StringTok{"DERS\_"}\NormalTok{, }\DecValTok{1}\SpecialCharTok{:}\DecValTok{16}\NormalTok{)}
\NormalTok{imp\_items }\OtherTok{\textless{}{-}} \FunctionTok{paste0}\NormalTok{(}\StringTok{"Hobbies\_Imp\_"}\NormalTok{, }\DecValTok{1}\SpecialCharTok{:}\DecValTok{8}\NormalTok{)}
\NormalTok{time\_items }\OtherTok{\textless{}{-}} \FunctionTok{paste0}\NormalTok{(}\StringTok{"Hobbies\_Time\_"}\NormalTok{, }\DecValTok{1}\SpecialCharTok{:}\DecValTok{8}\NormalTok{)}

\NormalTok{Data\_filtered }\OtherTok{\textless{}{-}}\NormalTok{ Data }\SpecialCharTok{\%\textgreater{}\%}
\FunctionTok{filter}\NormalTok{(Catch\_question }\SpecialCharTok{!=} \StringTok{"NA"}\NormalTok{) }\SpecialCharTok{\%\textgreater{}\%}
\FunctionTok{mutate}\NormalTok{(}
\AttributeTok{DERS\_mean =} \FunctionTok{rowMeans}\NormalTok{(}\FunctionTok{select}\NormalTok{(., }\FunctionTok{all\_of}\NormalTok{(ders\_items)), }\AttributeTok{na.rm =} \ConstantTok{TRUE}\NormalTok{),}
\AttributeTok{Imp\_overall =} \FunctionTok{rowMeans}\NormalTok{(}\FunctionTok{select}\NormalTok{(., }\FunctionTok{all\_of}\NormalTok{(imp\_items)), }\AttributeTok{na.rm =} \ConstantTok{TRUE}\NormalTok{),}
\AttributeTok{Time\_overall =} \FunctionTok{rowMeans}\NormalTok{(}\FunctionTok{select}\NormalTok{(., }\FunctionTok{all\_of}\NormalTok{(time\_items)), }\AttributeTok{na.rm =} \ConstantTok{TRUE}\NormalTok{)}
\NormalTok{)}

\NormalTok{Data\_analysis }\OtherTok{\textless{}{-}}\NormalTok{ Data\_filtered }\SpecialCharTok{\%\textgreater{}\%} \FunctionTok{filter}\NormalTok{(}\SpecialCharTok{!}\FunctionTok{is.na}\NormalTok{(DERS\_mean))}
\end{Highlighting}
\end{Shaded}

\subsubsection{Screening All Hobby
Variables}\label{screening-all-hobby-variables-1}

\begin{Shaded}
\begin{Highlighting}[]
\NormalTok{all\_hobby\_vars }\OtherTok{\textless{}{-}} \FunctionTok{c}\NormalTok{(imp\_items, time\_items, }\StringTok{"Imp\_overall"}\NormalTok{, }\StringTok{"Time\_overall"}\NormalTok{)}

\NormalTok{hobby\_table }\OtherTok{\textless{}{-}} \FunctionTok{tibble}\NormalTok{(}
\AttributeTok{Variable =}\NormalTok{ all\_hobby\_vars,}
\AttributeTok{Mean =} \FunctionTok{sapply}\NormalTok{(Data\_analysis[all\_hobby\_vars], mean, }\AttributeTok{na.rm =} \ConstantTok{TRUE}\NormalTok{),}
\AttributeTok{SD =} \FunctionTok{sapply}\NormalTok{(Data\_analysis[all\_hobby\_vars], sd, }\AttributeTok{na.rm =} \ConstantTok{TRUE}\NormalTok{),}
\AttributeTok{n =} \FunctionTok{sapply}\NormalTok{(Data\_analysis[all\_hobby\_vars], }\ControlFlowTok{function}\NormalTok{(x) }\FunctionTok{sum}\NormalTok{(}\SpecialCharTok{!}\FunctionTok{is.na}\NormalTok{(x)))}
\NormalTok{)}

\FunctionTok{kable}\NormalTok{(hobby\_table, }\AttributeTok{digits =} \DecValTok{3}\NormalTok{,}
\AttributeTok{caption =} \StringTok{"Means and Standard Deviations for All Hobby Variables"}\NormalTok{)}
\end{Highlighting}
\end{Shaded}

\begin{longtable}[]{@{}lrrr@{}}
\caption{Means and Standard Deviations for All Hobby
Variables}\tabularnewline
\toprule\noalign{}
Variable & Mean & SD & n \\
\midrule\noalign{}
\endfirsthead
\toprule\noalign{}
Variable & Mean & SD & n \\
\midrule\noalign{}
\endhead
\bottomrule\noalign{}
\endlastfoot
Hobbies\_Imp\_1 & 2.267 & 1.307 & 748 \\
Hobbies\_Imp\_2 & 2.116 & 1.091 & 748 \\
Hobbies\_Imp\_3 & 2.586 & 1.111 & 748 \\
Hobbies\_Imp\_4 & 3.229 & 1.017 & 748 \\
Hobbies\_Imp\_5 & 2.961 & 1.115 & 748 \\
Hobbies\_Imp\_6 & 4.167 & 0.798 & 748 \\
Hobbies\_Imp\_7 & 2.418 & 0.989 & 748 \\
Hobbies\_Imp\_8 & 3.270 & 1.063 & 748 \\
Hobbies\_Time\_1 & 1.717 & 1.131 & 748 \\
Hobbies\_Time\_2 & 1.509 & 0.925 & 748 \\
Hobbies\_Time\_3 & 2.028 & 1.293 & 748 \\
Hobbies\_Time\_4 & 4.162 & 1.388 & 748 \\
Hobbies\_Time\_5 & 1.999 & 1.336 & 748 \\
Hobbies\_Time\_6 & 5.352 & 1.501 & 748 \\
Hobbies\_Time\_7 & 1.488 & 0.840 & 748 \\
Hobbies\_Time\_8 & 2.485 & 1.370 & 748 \\
Imp\_overall & 2.877 & 0.468 & 748 \\
Time\_overall & 2.592 & 0.495 & 748 \\
\end{longtable}

\begin{Shaded}
\begin{Highlighting}[]
\NormalTok{screen\_results }\OtherTok{\textless{}{-}} \FunctionTok{tibble}\NormalTok{(}
\AttributeTok{Variable =}\NormalTok{ all\_hobby\_vars,}
\AttributeTok{r =} \ConstantTok{NA\_real\_}\NormalTok{,}
\AttributeTok{p\_value =} \ConstantTok{NA\_real\_}\NormalTok{,}
\AttributeTok{n =} \ConstantTok{NA\_integer\_}
\NormalTok{)}

\ControlFlowTok{for}\NormalTok{ (i }\ControlFlowTok{in} \FunctionTok{seq\_along}\NormalTok{(all\_hobby\_vars)) \{}
\NormalTok{v }\OtherTok{\textless{}{-}}\NormalTok{ all\_hobby\_vars[i]}
\NormalTok{tmp }\OtherTok{\textless{}{-}}\NormalTok{ Data\_analysis[, }\FunctionTok{c}\NormalTok{(}\StringTok{"DERS\_mean"}\NormalTok{, v)]}
\NormalTok{tmp }\OtherTok{\textless{}{-}}\NormalTok{ tmp[}\FunctionTok{complete.cases}\NormalTok{(tmp), ]}
\NormalTok{test }\OtherTok{\textless{}{-}} \FunctionTok{cor.test}\NormalTok{(tmp}\SpecialCharTok{$}\NormalTok{DERS\_mean, tmp[[v]])}

\NormalTok{screen\_results}\SpecialCharTok{$}\NormalTok{r[i] }\OtherTok{\textless{}{-}} \FunctionTok{unname}\NormalTok{(test}\SpecialCharTok{$}\NormalTok{estimate)}
\NormalTok{screen\_results}\SpecialCharTok{$}\NormalTok{p\_value[i] }\OtherTok{\textless{}{-}}\NormalTok{ test}\SpecialCharTok{$}\NormalTok{p.value}
\NormalTok{screen\_results}\SpecialCharTok{$}\NormalTok{n[i] }\OtherTok{\textless{}{-}} \FunctionTok{nrow}\NormalTok{(tmp)}
\NormalTok{\}}

\NormalTok{screen\_results }\OtherTok{\textless{}{-}}\NormalTok{ screen\_results }\SpecialCharTok{\%\textgreater{}\%}
\FunctionTok{mutate}\NormalTok{(}\AttributeTok{abs\_r =} \FunctionTok{abs}\NormalTok{(r)) }\SpecialCharTok{\%\textgreater{}\%}
\FunctionTok{arrange}\NormalTok{(}\FunctionTok{desc}\NormalTok{(abs\_r))}

\FunctionTok{kable}\NormalTok{(screen\_results,}
\AttributeTok{digits =} \DecValTok{3}\NormalTok{,}
\AttributeTok{caption =} \StringTok{"Correlations Between DERS\_mean and All Hobby Variables (Screening Table)"}\NormalTok{)}
\end{Highlighting}
\end{Shaded}

\begin{longtable}[]{@{}lrrrr@{}}
\caption{Correlations Between DERS\_mean and All Hobby Variables
(Screening Table)}\tabularnewline
\toprule\noalign{}
Variable & r & p\_value & n & abs\_r \\
\midrule\noalign{}
\endfirsthead
\toprule\noalign{}
Variable & r & p\_value & n & abs\_r \\
\midrule\noalign{}
\endhead
\bottomrule\noalign{}
\endlastfoot
Hobbies\_Time\_1 & -0.120 & 0.001 & 748 & 0.120 \\
Hobbies\_Imp\_4 & 0.108 & 0.003 & 748 & 0.108 \\
Hobbies\_Imp\_1 & -0.067 & 0.065 & 748 & 0.067 \\
Hobbies\_Imp\_8 & 0.067 & 0.066 & 748 & 0.067 \\
Hobbies\_Time\_4 & 0.063 & 0.084 & 748 & 0.063 \\
Imp\_overall & 0.050 & 0.174 & 748 & 0.050 \\
Hobbies\_Imp\_3 & 0.046 & 0.206 & 748 & 0.046 \\
Time\_overall & -0.034 & 0.357 & 748 & 0.034 \\
Hobbies\_Time\_2 & -0.027 & 0.466 & 748 & 0.027 \\
Hobbies\_Time\_5 & -0.025 & 0.488 & 748 & 0.025 \\
Hobbies\_Time\_6 & -0.021 & 0.564 & 748 & 0.021 \\
Hobbies\_Imp\_2 & 0.019 & 0.608 & 748 & 0.019 \\
Hobbies\_Imp\_6 & 0.017 & 0.643 & 748 & 0.017 \\
Hobbies\_Imp\_5 & 0.011 & 0.767 & 748 & 0.011 \\
Hobbies\_Time\_7 & 0.011 & 0.770 & 748 & 0.011 \\
Hobbies\_Time\_8 & -0.005 & 0.884 & 748 & 0.005 \\
Hobbies\_Imp\_7 & -0.005 & 0.896 & 748 & 0.005 \\
Hobbies\_Time\_3 & 0.003 & 0.944 & 748 & 0.003 \\
\end{longtable}

\begin{Shaded}
\begin{Highlighting}[]
\NormalTok{screen\_focus }\OtherTok{\textless{}{-}}\NormalTok{ screen\_results }\SpecialCharTok{\%\textgreater{}\%}
\FunctionTok{filter}\NormalTok{(}\SpecialCharTok{!}\FunctionTok{is.na}\NormalTok{(r),}
\NormalTok{abs\_r }\SpecialCharTok{\textgreater{}=} \FloatTok{0.10}\NormalTok{,}
\NormalTok{p\_value }\SpecialCharTok{\textless{}} \FloatTok{0.05}\NormalTok{)}

\FunctionTok{kable}\NormalTok{(screen\_focus,}
\AttributeTok{digits =} \DecValTok{3}\NormalTok{,}
\AttributeTok{caption =} \StringTok{"Hobby Variables With Significant Correlation With DERS\_mean"}\NormalTok{)}
\end{Highlighting}
\end{Shaded}

\begin{longtable}[]{@{}lrrrr@{}}
\caption{Hobby Variables With Significant Correlation With
DERS\_mean}\tabularnewline
\toprule\noalign{}
Variable & r & p\_value & n & abs\_r \\
\midrule\noalign{}
\endfirsthead
\toprule\noalign{}
Variable & r & p\_value & n & abs\_r \\
\midrule\noalign{}
\endhead
\bottomrule\noalign{}
\endlastfoot
Hobbies\_Time\_1 & -0.120 & 0.001 & 748 & 0.120 \\
Hobbies\_Imp\_4 & 0.108 & 0.003 & 748 & 0.108 \\
\end{longtable}

\subsubsection{Correlation Tests}\label{correlation-tests-1}

\begin{Shaded}
\begin{Highlighting}[]
\NormalTok{cor\_time1 }\OtherTok{\textless{}{-}} \FunctionTok{cor.test}\NormalTok{(}\SpecialCharTok{\textasciitilde{}}\NormalTok{ DERS\_mean }\SpecialCharTok{+}\NormalTok{ Hobbies\_Time\_1, }\AttributeTok{data =}\NormalTok{ Data\_analysis)}
\NormalTok{cor\_imp4  }\OtherTok{\textless{}{-}} \FunctionTok{cor.test}\NormalTok{(}\SpecialCharTok{\textasciitilde{}}\NormalTok{ DERS\_mean }\SpecialCharTok{+}\NormalTok{ Hobbies\_Imp\_4, }\AttributeTok{data =}\NormalTok{ Data\_analysis)}

\NormalTok{cor\_time1}
\end{Highlighting}
\end{Shaded}

\begin{verbatim}
## 
##  Pearson's product-moment correlation
## 
## data:  DERS_mean and Hobbies_Time_1
## t = -3.3048, df = 746, p-value = 0.0009959
## alternative hypothesis: true correlation is not equal to 0
## 95 percent confidence interval:
##  -0.1901676 -0.0488571
## sample estimates:
##        cor 
## -0.1201208
\end{verbatim}

\begin{Shaded}
\begin{Highlighting}[]
\NormalTok{cor\_imp4}
\end{Highlighting}
\end{Shaded}

\begin{verbatim}
## 
##  Pearson's product-moment correlation
## 
## data:  DERS_mean and Hobbies_Imp_4
## t = 2.969, df = 746, p-value = 0.003083
## alternative hypothesis: true correlation is not equal to 0
## 95 percent confidence interval:
##  0.03666589 0.17836881
## sample estimates:
##       cor 
## 0.1080662
\end{verbatim}

\subsubsection{Linear Models}\label{linear-models-1}

\begin{Shaded}
\begin{Highlighting}[]
\NormalTok{lm\_time1 }\OtherTok{\textless{}{-}} \FunctionTok{lm}\NormalTok{(DERS\_mean }\SpecialCharTok{\textasciitilde{}}\NormalTok{ Hobbies\_Time\_1, }\AttributeTok{data =}\NormalTok{ Data\_analysis)}
\NormalTok{lm\_imp4 }\OtherTok{\textless{}{-}} \FunctionTok{lm}\NormalTok{(DERS\_mean }\SpecialCharTok{\textasciitilde{}}\NormalTok{ Hobbies\_Imp\_4, }\AttributeTok{data =}\NormalTok{ Data\_analysis)}

\FunctionTok{summary}\NormalTok{(lm\_time1)}
\end{Highlighting}
\end{Shaded}

\begin{verbatim}
## 
## Call:
## lm(formula = DERS_mean ~ Hobbies_Time_1, data = Data_analysis)
## 
## Residuals:
##     Min      1Q  Median      3Q     Max 
## -1.9011 -0.7136 -0.0261  0.6964  2.1840 
## 
## Coefficients:
##                Estimate Std. Error t value Pr(>|t|)    
## (Intercept)     3.06112    0.06065  50.471  < 2e-16 ***
## Hobbies_Time_1 -0.09752    0.02951  -3.305 0.000996 ***
## ---
## Signif. codes:  0 '***' 0.001 '**' 0.01 '*' 0.05 '.' 0.1 ' ' 1
## 
## Residual standard error: 0.9123 on 746 degrees of freedom
## Multiple R-squared:  0.01443,    Adjusted R-squared:  0.01311 
## F-statistic: 10.92 on 1 and 746 DF,  p-value: 0.0009959
\end{verbatim}

\begin{Shaded}
\begin{Highlighting}[]
\FunctionTok{anova}\NormalTok{(lm\_time1)}
\end{Highlighting}
\end{Shaded}

\begin{verbatim}
## Analysis of Variance Table
## 
## Response: DERS_mean
##                 Df Sum Sq Mean Sq F value    Pr(>F)    
## Hobbies_Time_1   1   9.09  9.0907  10.922 0.0009959 ***
## Residuals      746 620.94  0.8324                      
## ---
## Signif. codes:  0 '***' 0.001 '**' 0.01 '*' 0.05 '.' 0.1 ' ' 1
\end{verbatim}

\begin{Shaded}
\begin{Highlighting}[]
\FunctionTok{summary}\NormalTok{(lm\_imp4)}
\end{Highlighting}
\end{Shaded}

\begin{verbatim}
## 
## Call:
## lm(formula = DERS_mean ~ Hobbies_Imp_4, data = Data_analysis)
## 
## Residuals:
##      Min       1Q   Median       3Q      Max 
## -2.00416 -0.71903 -0.03153  0.69110  2.22623 
## 
## Coefficients:
##               Estimate Std. Error t value Pr(>|t|)    
## (Intercept)    2.57850    0.11130  23.167  < 2e-16 ***
## Hobbies_Imp_4  0.09763    0.03288   2.969  0.00308 ** 
## ---
## Signif. codes:  0 '***' 0.001 '**' 0.01 '*' 0.05 '.' 0.1 ' ' 1
## 
## Residual standard error: 0.9136 on 746 degrees of freedom
## Multiple R-squared:  0.01168,    Adjusted R-squared:  0.01035 
## F-statistic: 8.815 on 1 and 746 DF,  p-value: 0.003083
\end{verbatim}

\begin{Shaded}
\begin{Highlighting}[]
\FunctionTok{anova}\NormalTok{(lm\_imp4)}
\end{Highlighting}
\end{Shaded}

\begin{verbatim}
## Analysis of Variance Table
## 
## Response: DERS_mean
##                Df Sum Sq Mean Sq F value   Pr(>F)   
## Hobbies_Imp_4   1   7.36  7.3576   8.815 0.003083 **
## Residuals     746 622.67  0.8347                    
## ---
## Signif. codes:  0 '***' 0.001 '**' 0.01 '*' 0.05 '.' 0.1 ' ' 1
\end{verbatim}

\subsubsection{Visualizations}\label{visualizations-1}

\begin{Shaded}
\begin{Highlighting}[]
\FunctionTok{plot}\NormalTok{(DERS\_mean }\SpecialCharTok{\textasciitilde{}}\NormalTok{ Hobbies\_Time\_1,}
\AttributeTok{data =}\NormalTok{ Data\_analysis,}
\AttributeTok{xlab =} \StringTok{"Hours Spent on Athletics (Hobbies\_Time\_1)"}\NormalTok{,}
\AttributeTok{ylab =} \StringTok{"DERS Mean"}\NormalTok{,}
\AttributeTok{main =} \StringTok{"DERS\_mean vs Hobbies\_Time\_1"}\NormalTok{)}

\FunctionTok{abline}\NormalTok{(lm\_time1, }\AttributeTok{col =} \StringTok{"red"}\NormalTok{, }\AttributeTok{lwd =} \DecValTok{2}\NormalTok{)}
\end{Highlighting}
\end{Shaded}

\includegraphics{final_analysis_files/figure-latex/unnamed-chunk-64-1.pdf}

\begin{Shaded}
\begin{Highlighting}[]
\FunctionTok{plot}\NormalTok{(DERS\_mean }\SpecialCharTok{\textasciitilde{}}\NormalTok{ Hobbies\_Imp\_4,}
\AttributeTok{data =}\NormalTok{ Data\_analysis,}
\AttributeTok{xlab =} \StringTok{"Importance of Recreational Streaming (Hobbies\_Imp\_4)"}\NormalTok{,}
\AttributeTok{ylab =} \StringTok{"DERS Mean"}\NormalTok{,}
\AttributeTok{main =} \StringTok{"DERS\_mean vs Hobbies\_Imp\_4"}\NormalTok{)}

\FunctionTok{abline}\NormalTok{(lm\_imp4, }\AttributeTok{col =} \StringTok{"red"}\NormalTok{, }\AttributeTok{lwd =} \DecValTok{2}\NormalTok{)}
\end{Highlighting}
\end{Shaded}

\includegraphics{final_analysis_files/figure-latex/unnamed-chunk-64-2.pdf}

\end{document}
